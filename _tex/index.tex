% Options for packages loaded elsewhere
\PassOptionsToPackage{unicode}{hyperref}
\PassOptionsToPackage{hyphens}{url}
\PassOptionsToPackage{dvipsnames,svgnames,x11names}{xcolor}
%
\documentclass[
  letterpaper,
  DIV=11,
  numbers=noendperiod]{scrartcl}

\usepackage{amsmath,amssymb}
\usepackage{iftex}
\ifPDFTeX
  \usepackage[T1]{fontenc}
  \usepackage[utf8]{inputenc}
  \usepackage{textcomp} % provide euro and other symbols
\else % if luatex or xetex
  \usepackage{unicode-math}
  \defaultfontfeatures{Scale=MatchLowercase}
  \defaultfontfeatures[\rmfamily]{Ligatures=TeX,Scale=1}
\fi
\usepackage{lmodern}
\ifPDFTeX\else  
    % xetex/luatex font selection
\fi
% Use upquote if available, for straight quotes in verbatim environments
\IfFileExists{upquote.sty}{\usepackage{upquote}}{}
\IfFileExists{microtype.sty}{% use microtype if available
  \usepackage[]{microtype}
  \UseMicrotypeSet[protrusion]{basicmath} % disable protrusion for tt fonts
}{}
\makeatletter
\@ifundefined{KOMAClassName}{% if non-KOMA class
  \IfFileExists{parskip.sty}{%
    \usepackage{parskip}
  }{% else
    \setlength{\parindent}{0pt}
    \setlength{\parskip}{6pt plus 2pt minus 1pt}}
}{% if KOMA class
  \KOMAoptions{parskip=half}}
\makeatother
\usepackage{xcolor}
\setlength{\emergencystretch}{3em} % prevent overfull lines
\setcounter{secnumdepth}{-\maxdimen} % remove section numbering
% Make \paragraph and \subparagraph free-standing
\makeatletter
\ifx\paragraph\undefined\else
  \let\oldparagraph\paragraph
  \renewcommand{\paragraph}{
    \@ifstar
      \xxxParagraphStar
      \xxxParagraphNoStar
  }
  \newcommand{\xxxParagraphStar}[1]{\oldparagraph*{#1}\mbox{}}
  \newcommand{\xxxParagraphNoStar}[1]{\oldparagraph{#1}\mbox{}}
\fi
\ifx\subparagraph\undefined\else
  \let\oldsubparagraph\subparagraph
  \renewcommand{\subparagraph}{
    \@ifstar
      \xxxSubParagraphStar
      \xxxSubParagraphNoStar
  }
  \newcommand{\xxxSubParagraphStar}[1]{\oldsubparagraph*{#1}\mbox{}}
  \newcommand{\xxxSubParagraphNoStar}[1]{\oldsubparagraph{#1}\mbox{}}
\fi
\makeatother


\providecommand{\tightlist}{%
  \setlength{\itemsep}{0pt}\setlength{\parskip}{0pt}}\usepackage{longtable,booktabs,array}
\usepackage{calc} % for calculating minipage widths
% Correct order of tables after \paragraph or \subparagraph
\usepackage{etoolbox}
\makeatletter
\patchcmd\longtable{\par}{\if@noskipsec\mbox{}\fi\par}{}{}
\makeatother
% Allow footnotes in longtable head/foot
\IfFileExists{footnotehyper.sty}{\usepackage{footnotehyper}}{\usepackage{footnote}}
\makesavenoteenv{longtable}
\usepackage{graphicx}
\makeatletter
\newsavebox\pandoc@box
\newcommand*\pandocbounded[1]{% scales image to fit in text height/width
  \sbox\pandoc@box{#1}%
  \Gscale@div\@tempa{\textheight}{\dimexpr\ht\pandoc@box+\dp\pandoc@box\relax}%
  \Gscale@div\@tempb{\linewidth}{\wd\pandoc@box}%
  \ifdim\@tempb\p@<\@tempa\p@\let\@tempa\@tempb\fi% select the smaller of both
  \ifdim\@tempa\p@<\p@\scalebox{\@tempa}{\usebox\pandoc@box}%
  \else\usebox{\pandoc@box}%
  \fi%
}
% Set default figure placement to htbp
\def\fps@figure{htbp}
\makeatother
% definitions for citeproc citations
\NewDocumentCommand\citeproctext{}{}
\NewDocumentCommand\citeproc{mm}{%
  \begingroup\def\citeproctext{#2}\cite{#1}\endgroup}
\makeatletter
 % allow citations to break across lines
 \let\@cite@ofmt\@firstofone
 % avoid brackets around text for \cite:
 \def\@biblabel#1{}
 \def\@cite#1#2{{#1\if@tempswa , #2\fi}}
\makeatother
\newlength{\cslhangindent}
\setlength{\cslhangindent}{1.5em}
\newlength{\csllabelwidth}
\setlength{\csllabelwidth}{3em}
\newenvironment{CSLReferences}[2] % #1 hanging-indent, #2 entry-spacing
 {\begin{list}{}{%
  \setlength{\itemindent}{0pt}
  \setlength{\leftmargin}{0pt}
  \setlength{\parsep}{0pt}
  % turn on hanging indent if param 1 is 1
  \ifodd #1
   \setlength{\leftmargin}{\cslhangindent}
   \setlength{\itemindent}{-1\cslhangindent}
  \fi
  % set entry spacing
  \setlength{\itemsep}{#2\baselineskip}}}
 {\end{list}}
\usepackage{calc}
\newcommand{\CSLBlock}[1]{\hfill\break\parbox[t]{\linewidth}{\strut\ignorespaces#1\strut}}
\newcommand{\CSLLeftMargin}[1]{\parbox[t]{\csllabelwidth}{\strut#1\strut}}
\newcommand{\CSLRightInline}[1]{\parbox[t]{\linewidth - \csllabelwidth}{\strut#1\strut}}
\newcommand{\CSLIndent}[1]{\hspace{\cslhangindent}#1}

\KOMAoption{captions}{tablesignature}
\makeatletter
\@ifpackageloaded{caption}{}{\usepackage{caption}}
\AtBeginDocument{%
\ifdefined\contentsname
  \renewcommand*\contentsname{Table of contents}
\else
  \newcommand\contentsname{Table of contents}
\fi
\ifdefined\listfigurename
  \renewcommand*\listfigurename{List of Figures}
\else
  \newcommand\listfigurename{List of Figures}
\fi
\ifdefined\listtablename
  \renewcommand*\listtablename{List of Tables}
\else
  \newcommand\listtablename{List of Tables}
\fi
\ifdefined\figurename
  \renewcommand*\figurename{Figure}
\else
  \newcommand\figurename{Figure}
\fi
\ifdefined\tablename
  \renewcommand*\tablename{Table}
\else
  \newcommand\tablename{Table}
\fi
}
\@ifpackageloaded{float}{}{\usepackage{float}}
\floatstyle{ruled}
\@ifundefined{c@chapter}{\newfloat{codelisting}{h}{lop}}{\newfloat{codelisting}{h}{lop}[chapter]}
\floatname{codelisting}{Listing}
\newcommand*\listoflistings{\listof{codelisting}{List of Listings}}
\makeatother
\makeatletter
\makeatother
\makeatletter
\@ifpackageloaded{caption}{}{\usepackage{caption}}
\@ifpackageloaded{subcaption}{}{\usepackage{subcaption}}
\makeatother

\usepackage{bookmark}

\IfFileExists{xurl.sty}{\usepackage{xurl}}{} % add URL line breaks if available
\urlstyle{same} % disable monospaced font for URLs
\hypersetup{
  pdftitle={Evaluating Cellular Communication Sensing for Lapse Risk Prediction During Early Recovery from Alcohol Use Disorder},
  pdfauthor={Kendra Wyant; Coco Yu; John J. Curtin},
  pdfkeywords={Substance use disorders, Machine learning, Cellular
Sensing},
  colorlinks=true,
  linkcolor={blue},
  filecolor={Maroon},
  citecolor={Blue},
  urlcolor={Blue},
  pdfcreator={LaTeX via pandoc}}


\title{Evaluating Cellular Communication Sensing for Lapse Risk
Prediction During Early Recovery from Alcohol Use Disorder}
\author{Kendra Wyant \and Coco Yu \and John J. Curtin}
\date{2026-01-11}

\begin{document}
\maketitle


\textsubscript{Source:
\href{https://jjcurtin.github.io/study_messages/index.qmd.html}{Article
Notebook}}

\section{Introduction}\label{introduction}

Alcohol Use Disorder (AUD) is a chronic, relapsing disease (Dennis \&
Scott, 2007; McLellan et al., 2000; Rounsaville, 2010). Lapses, single
episodes of alcohol use, are among the strongest predictors (and a
necessary precursor) for relapse, a full return to harmful drinking
(Marlatt \& Donovan, 2007; Marlatt \& Gordon, 1985). While lapses can
occur at any point in recovery, they are particularly risky during early
recovery (Daley \& Douaihy, 2019). Protective coping mechanisms and
socio-environmental resources that support recovery are dynamic and
accumulate over time (Cleveland et al., 2021). As a result, early
recovery represents a critical window of vulnerability during which a
lapse is more likely to escalate into relapse.

An automated recovery support system powered by personal sensing and
machine learning may assist with the inherently difficult task of
identifying when and why someone is at increased risk for lapse.
Personal sensing of densely sampled data from individuals' day-to-day
lives can provide the inputs necessary for temporally dynamic lapse
predictions (Mohr et al., 2017). Early machine learning models using
ecological momentary assessment data have achieved excellent accuracy
(Chih et al., 2014; Wyant et al., 2024, under review).

Still, questions remain about the long-term feasibility of self-report
sensing methods and whether new, important risk factors might emerge
from sensing methods that passively collect smartphone data without user
input.

Cellular communication sensing may be one promising method. It offers
the potential for greater temporal specificity for capturing
fluctuations in risk compared with self-report data (i.e., as it occurs
vs.~prompting users to make reports the next day). Late night phone
calls could indicate an emergency, ``drunk dialing'', or interpersonal
conflict. It can also capture risk-relevant constructs difficult for
people to self report, such as slow changing social patterns. A decrease
in the number of contacts someone is communicating with could indicate a
shrinking social circle, isolation, or disengagement.

These passive data may become more powerful still when communications
are contextualized with personal meaning for a given participant (e.g.,
Who is this contact to them?). In the above examples, contextualized
communication data might reveal that late-night phone calls are to a
sponsor or that a shrinking social circle is due to reduced contact with
people unsupportive of their recovery.

In this study, we evaluated the performance of a machine learning model
that predicts the probability of a next-day alcohol lapse among
individuals in early recovery from AUD using contextualized cellular
communication data. We also describe the most important features
contributing to these predictions, with the goal of identifying new,
clinically meaningful features emerging from communication-based
sensing.

\section{Methods}\label{methods}

\subsection{Participants and
Procedure}\label{participants-and-procedure}

We recruited adults in early recovery from AUD in Madison, Wisconsin,
through print and digital advertisements and partnerships with treatment
centers. Eligibility criteria required that participants were age 18 or
older, able to read and write in English, had moderate to severe AUD
\footnote{(≥4 self-reported DSM-5 symptoms)}, had a goal of abstinence
from alcohol, had been abstinent for 1--8 weeks, were willing to use a
single smartphone, and were not exhibiting severe psychosis or
paranoia.\footnote{Defined as scores \textgreater2.2 or 2.8,
  respectively, on the psychosis or paranoia scales of the Symptom
  Checklist--90 (Derogatis, L.R., 2000).}

Participants completed up to 5 study visits over approximately 3 months:
a screening visit, intake visit, and 3 monthly follow-up visits. At
screening we collected demographic information (age, sex at birth, race,
ethnicity, education, marital status, employment, and income) and
clinical characteristics (DSM-5 AUD symptom count, alcohol problems
(Hurlbut \& Sher, 1992), and presence of psychological symptoms
(Derogatis, L.R., 2000)). At intake we collected additional self-report
data on abstinence self-efficacy (McKiernan et al., 2011), craving
(Flannery et al., 1999), and recent recovery efforts.

At each monthly follow-up, we downloaded backups of participants'
cellular communication metadata directly from their smartphones.
Metadata included the phone number of the other party, the date and time
of the communication, the origin of call or message (i.e., incoming or
outgoing), whether the call was answered (voice calls only), and the
duration of the call (voice calls only). During each follow-up visit,
study staff identified important contacts. Contacts that participants
communicated with at least twice by call or text in the past month were
deemed important. For each important contact, participants answered 7
contextual questions about their type of relationship, whether they ever
drank alcohol with this person, the drinking status of the contact,
expectations about whether the contact would drink in their presence,
recovery status of contact, level of supportiveness of contact, and
affective experiences with the contact.

While enrolled, participants completed 4 brief daily ecological
momentary assessments (7-10 questions). The first item assessed alcohol
use (date and time of any unreported drinking episodes). Lapse reports
were verified at follow-up visits using a timeline follow-back
interview. Additional sensing data streams and self-report measures were
collected for the parent grant. The full study protocol is available on
our Open Science Framework page (\url{https://osf.io/wgpz9/}).

\subsection{Data Analysis Plan}\label{data-analysis-plan}

Our models predicted the probability of an alcohol lapse within a
24-hour window. Predictions were generated daily at 4 a.m., beginning on
participants' second study day and continuing for up to 3 months.
Participants reported the date and hour of the start and end time of any
alcohol use on the first item of the EMA. Prediction windows were
labeled as lapse if any alcohol use was reported in the 24-hour window.
In total, there were 11,507 labeled prediction windows across all
participants. Positive lapse labels were underrepresented (7.5\%;
861/11,507).

We filtered the data to include only communications with known context
(i.e., people with whom they communicated with at least twice in a month
and whom they provided self-report context about). Cellular
communication features were engineered from all available data up to the
start of each window. We used six scoring epochs (6, 12, 24, 48, 72, and
168 hours before the start of the prediction window) to create features.
Within each scoring epoch we calculated two types of features: raw and
difference features. Raw features represent the raw feature value
calculated within a scoring epoch (e.g., the rate count of text messages
during the 48 hours immediately preceding the start of the prediction
window). Difference features capture participant-level changes from
their baseline scores. Specifically, we subtracted each participant's
mean score for each feature (using all available data prior to the
prediction window) from the associated raw feature (e.g., the
participant's average rate count of text messages across all time on
study subtracted from the rate count in the preceding 48 hours).

The full model included 406 features from cellular communication data
plus 24 numeric or dummy-coded features from baseline self-report
measures. We also evaluated a comparison model that used only the
baseline features. Table~\ref{tbl-1} details the raw predictors, feature
engineering procedures, and features included in the full vs.~baseline
models. Other feature engineering steps performed during
cross-validation included imputing missing values (median imputation for
numeric features and mode imputation for nominal features) and removing
zero and near-zero variance features as determined from held-in data.

Candidate model configurations differed by algorithm (elastic net,
random forest, XGBoost), outcome resampling method, and hyperparameter
values. The best configuration for each model was selected using 6
repeats of participant-grouped 5-fold cross-validation. Our performance
metric was area under the receiver operating curve (auROC). Folds were
stratified so that all folds contained comparable proportions of
individuals who lapsed frequently (i.e., 10+ times).

We evaluated model performance with a Bayesian hierarchical generalized
linear model. Posterior distributions with 95\% credible intervals (CI)
were estimated from the 30 held-out test sets using weakly informative,
data-dependent priors to regularize and reduce overfitting.\footnote{Residual
  SD \textasciitilde{} normal(0, exp(2)); intercept (centered
  predictors) \textasciitilde{} normal(2.3, 1.3); window-width contrasts
  \textasciitilde{} normal(0, 2.69); covariance \textasciitilde{}
  decov(1,1,1,1).} Random intercepts were included for repeat and fold
(nested within repeat). auROCs were logit-transformed and regressed on
model type to estimate the probability that model performances differed
systematically.

Our best performing models used an elastic net algorithm. We quantified
feature importance by examining the retained features (i.e., coefficient
value \textgreater{} 0) in the full model and ordering them by absolute
coefficient value. These values provide an estimate of the direction and
magnitude of association between each predictor and the outcome,
conditional on the other features retained. All our annotated analysis
scripts are publicly available on our study website
(\url{https://jjcurtin.github.io/study_messages/}).

\begin{longtable}[]{@{}llllrll@{}}

\toprule\noalign{}
Raw Predictor & Response Options & Feature Engineering & Scoring Epochs
& Total Features & Full Model & Baseline Model \\
\midrule\noalign{}
\endhead
\bottomrule\noalign{}
\endlastfoot
Originated & Incoming, outgoing & Difference and raw rate counts for
text messages and voice calls & 6, 12, 24, 48, 72, and 168 hours & 48 &
Yes & No \\
Call duration & Duration (in minutes) & Difference and raw rate sums of
duration, difference and raw most recent duration & 6, 12, 24, 48, 72,
and 168 hours & 14 & Yes & No \\
Call answered & Yes, no & Difference and raw rate counts for unanswered
incoming voice calls & 6, 12, 24, 48, 72, and 168 hours & 12 & Yes &
No \\
Date/time of communication & Date and time & Difference and raw rate
counts for text messages and voice calls at night (10 pm -- 6am) and on
weekends & 24, 48, 72, and 168 hours (night), 168 hours (weekend) & 20 &
Yes & No \\
Phone number & Phone number & Difference and raw rate counts of unique
phone numbers & 6, 12, 24, 48, 72, and 168 hours & 12 & Yes & No \\
Type of Relationship & Family, friend, counselor or social worker,
co-worker & Difference and raw rate counts of unique phone numbers & 6,
12, 24, 48, 72, and 168 hours & 48 & Yes & No \\
Have you drank alcohol with this person? & Never/almost never,
occasionally, almost always/always & Difference and raw rate counts of
each response option & 6, 12, 24, 48, 72, and 168 hours & 36 & Yes &
No \\
What is their drinking status? & Drinker, non-drinker, don't know &
Difference and raw rate counts of each response option & 6, 12, 24, 48,
72, and 168 hours & 36 & Yes & No \\
Would you expect them to drink in your presence? & Yes, no, uncertain &
Difference and raw rate counts of each response option & 6, 12, 24, 48,
72, and 168 hours & 36 & Yes & No \\
Are they currently in recovery from drugs or alcohol? & Yes, no, don't
know & Difference and raw rate counts of each response option & 6, 12,
24, 48, 72, and 168 hours & 36 & Yes & No \\
Are they supportive about your recovery goals? & Supportive,
unsupportive, mixed, neutral, don't know & Difference and raw rate
counts of each response option & 6, 12, 24, 48, 72, and 168 hours & 60 &
Yes & No \\
How are your typical experiences with this person? & Pleasant,
unpleasant, mixed, neutral & Difference and raw rate counts of each
response option & 6, 12, 24, 48, 72, and 168 hours & 48 & Yes & No \\
DSM-5 symptom count & Numeric (4-11) & & & 1 & Yes & Yes \\
Past year alcohol problems & Numeric (0-27) & & & 1 & Yes & Yes \\
Craving & Numeric (0-30) & & & 1 & Yes & Yes \\
Abstinence self-efficacy: Negative affect, social, physical, and craving
subscales & Numeric (0-20) & & & 4 & Yes & Yes \\
Number of individual alcohol counseling sessions attended (past 30 days)
& Numeric & & & 1 & Yes & Yes \\
Number of group alcohol counseling sessions attended (past 30 days) &
Numeric & & & 1 & Yes & Yes \\
Number of self-help group meetings attended (past 30 days) & Numeric & &
& 1 & Yes & Yes \\
Number of other mental health counseling sessions attended (past 30
days) & Numeric & & & 1 & Yes & Yes \\
Number of days in contact with supportive people (past 30 days) &
Numeric & & & 1 & Yes & Yes \\
Number of days in contact with unsupportive people (past 30 days) &
Numeric & & & 1 & Yes & Yes \\
Taken prescribed medication for alcohol use disorder (past 30 days) &
Yes, no & Dummy coded & & 1 & Yes & Yes \\
Taken prescribed medication for other mental health disorder (past 30
days) & Yes, no & Dummy coded & & 1 & Yes & Yes \\
Satisfaction with progress toward recovery goals (past 30 days) &
Numeric (0-4) & & & 1 & Yes & Yes \\
Confidence in abstinence ability (next 30 days) & Numeric (0-4) & & & 1
& Yes & Yes \\
Has a goal of abstinence & Yes, no, uncertain & Dummy coded & & 2 & Yes
& Yes \\
Age & Numeric (years) & & & 1 & Yes & Yes \\
Sex at birth & Male, female & Dummy coded & & 1 & Yes & Yes \\
Race & Non-Hispanic White, non-White and/or Hispanic & Dummy coded & & 1
& Yes & Yes \\
Education & High school or less, some college, college degree & Dummy
coded & & 2 & Yes & Yes \\
Income & Numeric (dollars) & & & 1 & Yes & Yes \\
Marital Status & Married, not married, other & Dummy coded & & 2 & Yes &
Yes \\


\caption{\label{tbl-1}Feature Engineering of Raw Predictors}

\tabularnewline
\end{longtable}

\textsubscript{Source:
\href{https://jjcurtin.github.io/study_messages/notebooks/mak_tables-preview.html\#cell-tbl-1}{Make
All Tables for Main Manuscript}}

\subsection{Ethical Considerations}\label{ethical-considerations}

All procedures were approved by the University of Wisconsin-Madison
Institutional Review Board (Study \#2015-0780). All participants
provided written informed consent.

\section{Results}\label{results}

\subsection{Participants}\label{participants}

We screened 192 participants. Of these, 169 enrolled and 154 completed
the first follow-up visit. Data from 1 participant was excluded due to
not reporting a goal of abstinence and lapsing multiple times a day
every day on study. Data from 1 participant was excluded due to evidence
of careless responding. Data from 1 participant was excluded due to poor
compliance with EMA resulting in questionable lapse labels. Data from 7
participants were excluded due to poor compliance providing
communication data (i.e., deleting logs prior to download and/or not
providing context information about important contacts). The final
analytic sample included 144 participants. Table~\ref{tbl-2} provides
the demographic characterization of our sample. 56\% of participants
reported at least one lapse while on study.

\begin{longtable}[]{@{}lrrlll@{}}

\toprule\noalign{}
& N & \% & M & SD & Range \\
\midrule\noalign{}
\endhead
\midrule\noalign{}
{Note: } & & & & & \\
\textsuperscript{} N = 144 & & & & & \\
\bottomrule\noalign{}
\endlastfoot
Age & & & 40.4 & 11.8 & 21-72 \\
Sex at Birth & & & & & \\
\multicolumn{6}{@{}l@{}}{%
\textbf{}} \\
Female & 74 & 51.4 & & & \\
Male & 70 & 48.6 & & & \\
Race & & & & & \\
\multicolumn{6}{@{}l@{}}{%
\textbf{}} \\
American Indian/Alaska Native & 3 & 2.1 & & & \\
Asian & 2 & 1.4 & & & \\
Black/African American & 8 & 5.6 & & & \\
White/Caucasian & 125 & 86.8 & & & \\
Other/Multiracial & 6 & 4.2 & & & \\
Hispanic, Latino, or Spanish origin & & & & & \\
\multicolumn{6}{@{}l@{}}{%
\textbf{}} \\
Yes & 3 & 2.1 & & & \\
No & 141 & 97.9 & & & \\
Education & & & & & \\
\multicolumn{6}{@{}l@{}}{%
\textbf{}} \\
Less than high school or GED degree & 1 & 0.7 & & & \\
High school or GED & 14 & 9.7 & & & \\
Some college & 39 & 27.1 & & & \\
2-Year degree & 13 & 9.0 & & & \\
College degree & 55 & 38.2 & & & \\
Advanced degree & 22 & 15.3 & & & \\
Employment & & & & & \\
\multicolumn{6}{@{}l@{}}{%
\textbf{}} \\
Employed full-time & 70 & 48.6 & & & \\
Employed part-time & 25 & 17.4 & & & \\
Full-time student & 7 & 4.9 & & & \\
Homemaker & 1 & 0.7 & & & \\
Disabled & 7 & 4.9 & & & \\
Retired & 8 & 5.6 & & & \\
Unemployed & 15 & 10.4 & & & \\
Temporarily laid off, sick leave, or maternity leave & 3 & 2.1 & & & \\
Other, not otherwise specified & 8 & 5.6 & & & \\
Personal Income & & & \$35,050 & \$32,069 & \$0-200,000 \\
Marital Status & & & & & \\
\multicolumn{6}{@{}l@{}}{%
\textbf{}} \\
Never married & 63 & 43.8 & & & \\
Married & 32 & 22.2 & & & \\
Divorced & 42 & 29.2 & & & \\
Separated & 5 & 3.5 & & & \\
Widowed & 2 & 1.4 & & & \\


\caption{\label{tbl-2}Demographics}

\tabularnewline
\end{longtable}

\textsubscript{Source:
\href{https://jjcurtin.github.io/study_messages/notebooks/mak_tables-preview.html\#cell-tbl-2}{Make
All Tables for Main Manuscript}}

\subsection{Communications}\label{communications}

Participants had an average of 26 important contacts (range 2-113) that
were contextualized with self-report information. We obtained a total of
375,912 contextualized communications across participants. Participants
had, on average, 2,610 contextualized communications (range =
109-14,225) averaging to about 33 communications per day (range 3-278).

\subsection{Model Evaluation}\label{model-evaluation}

The median posterior auROC for the full model was 0.68, with relatively
narrow 95\% CI ({[}0.64, 0.71{]}) that did not contain .5. This provides
strong evidence that the model is capturing signal in the data. The
final model retained 13 features (Figure~\ref{fig-1}). The top four were
baseline measures of abstinence confidence, having a goal of abstinence,
abstinence self-efficacy when experiencing negative affect, and craving.
Communication frequency with people unaware of the individual's recovery
goals also emerged as an important feature associated with increased
lapse risk.

We evaluated a comparison model to assess the incremental predictive
value of cellular communication features beyond baseline measures. The
baseline model retained 5 features and achieved performance nearly
identical to the full model (median auROC = 0.68, 95\% CI {[}0.64,
0.71{]}). The median difference in auROC between the full and baseline
models was less than .01, providing no evidence (52\% probability) that
the full model performed better than the baseline model.

\begin{figure}[H]

\centering{

\pandocbounded{\includegraphics[keepaspectratio]{index_files/figure-latex/notebooks-mak_figures-fig-1-output-1.png}}

}

\caption{\label{fig-1}Global feature importance (elastic net
coefficient) for the full model. Features are ordered by absolute
coefficient value. Blue bars indicate higher feature values, on average,
lower lapse risk. Red bars indicate higher feature values, on average,
increase risk. Baseline features were collected from self-report
measures at the start of the study. Communication features were
engineered from the contexualized cellular communications.}

\end{figure}%

\textsubscript{Source:
\href{https://jjcurtin.github.io/study_messages/notebooks/mak_figures-preview.html\#cell-fig-1}{Make
All Figures for Main Manuscript}}

\section{Discussion}\label{discussion}

Our model achieved fair performance, with an auROC of 0.68, indicating
that some predictive signal was present. However, it did not offer
incremental value beyond a baseline model that included only demographic
and self-report measures. Consistent with this, the four most important
predictors in our model were all self-report variables: abstinence
confidence, abstinence goal, negative affect efficacy, and craving.

Nonetheless, several communication features were retained in the final
model with moderately sized coefficients. These included communications
with people unaware of the participant's recovery status, non-drinkers,
friends, and individuals who were unpleasant to interact with. In
contrast, raw counts of calls and text messages and call durations were
not retained in the final model. This implies that the quantity of
communication may be less informative than the quality and social
significance. Future research may benefit from collecting richer
contextual data about communication contacts to better understand the
social dynamics contributing to lapse risk.

Even with highly contextualized communication data, however, prediction
may be limited by data sparsity. Many participants had few daily
communications, and some had extended periods with no recorded
interactions at all. Our study design may have further contributed to
this limitation. We collected only phone and SMS text communications
through the native smartphone app. In recent years, many individuals use
private messaging apps (e.g., WhatsApp, Signal) or social media
platforms (e.g., Facebook Messenger, Instagram) as their primary
communication method (McDowell et al., 2025). Therefore, our dataset
likely missed a substantial portion of participants' communications.
Notably, the communication features that were retained in the final
model were scored over our longest scoring epochs (72 and 168 hours),
suggesting that with more communications these features may be become
more important. Future studies could explore whether increasing
communication data from these alternative platforms yield stronger
predictive signals.

We cannot entirely dismiss the potential value of cellular communication
data for risk prediction. For example, researchers have successfully
incorporated communication data into models with other sensing data
(e.g., accelerometer, geolocation, and device usage) to predict alcohol
use episodes (S. Bae et al., 2017; S. W. Bae et al., 2023). However,
even in these instances, the contribution of cellular communications is
questionable and other sensing methods like geolocation appear to be
more promising . Other practical challenges in collecting call and text
message data further limit the feasibility of this sensing method. For
example, we obtained participants' cellular communication data by
downloading backups of their communication logs in person during their
monthly follow-up visits. However, Apple heavily restricts apps in its
app store from accessing call and text message data, making real-time
sensing of communications challenging (if not impossible) for IOS users.
We conclude that other forms of social interaction characterization
(e.g., engineering time spent with supportive contacts from geolocation
data) are more worthwhile to pursue in future research.

\newpage

\phantomsection\label{refs}
\begin{CSLReferences}{1}{0}
\bibitem[\citeproctext]{ref-baeLeveragingMobilePhone2023}
Bae, S. W., Suffoletto, B., Zhang, T., Chung, T., Ozolcer, M., Islam, M.
R., \& Dey, A. (2023). Leveraging {Mobile Phone Sensors}, {Machine
Learning} and {Explainable Artificial Intelligence} to {Predict Imminent
Same-Day Binge Drinking Events} to {Support Just-In-Time Adaptive
Interventions}: {A Feasibility Study}. \emph{JMIR Formative Research}.
\url{https://doi.org/10.2196/39862}

\bibitem[\citeproctext]{ref-baeDetectingDrinkingEpisodes2017}
Bae, S., Ferreira, D., Suffoletto, B., Puyana, J. C., Kurtz, R., Chung,
T., \& Dey, A. K. (2017). Detecting {Drinking Episodes} in {Young Adults
Using Smartphone-based Sensors}. \emph{Proceedings of the ACM on
Interactive, Mobile, Wearable and Ubiquitous Technologies}, \emph{1}(2),
1--36. \url{https://doi.org/10.1145/3090051}

\bibitem[\citeproctext]{ref-chihPredictiveModelingAddiction2014}
Chih, M.-Y., Patton, T., McTavish, F. M., Isham, A. J., Judkins-Fisher,
C. L., Atwood, A. K., \& Gustafson, D. H. (2014). Predictive modeling of
addiction lapses in a mobile health application. \emph{Journal of
Substance Abuse Treatment}, \emph{46}(1), 29--35.
\url{https://doi.org/10.1016/j.jsat.2013.08.004}

\bibitem[\citeproctext]{ref-clevelandRecoveryRecoveryCapital2021}
Cleveland, H. H., Brick, T. R., Knapp, K. S., \& Croff, J. M. (2021).
Recovery and {Recovery Capital}: {Aligning Measurement} with {Theory}
and {Practice}. In J. M. Croff \& J. Beaman (Eds.), \emph{Family
{Resilience} and {Recovery} from {Opioids} and {Other Addictions}} (pp.
109--128). Springer International Publishing.
\url{https://doi.org/10.1007/978-3-030-56958-7_6}

\bibitem[\citeproctext]{ref-daleyReducingRiskRelapse2019}
Daley, D. C., \& Douaihy, A. (2019). Reducing the {Risk} of {Relapse}.
In D. C. Daley, A. B. Douaihy, D. C. Daley, \& A. Douaihy (Eds.),
\emph{Managing {Substance Use Disorder}: {Practitioner Guide}} (p. 0).
Oxford University Press.
\url{https://doi.org/10.1093/med-psych/9780190926717.003.0018}

\bibitem[\citeproctext]{ref-dennisManagingAddictionChronic2007}
Dennis, M., \& Scott, C. K. (2007).
\href{https://www.ncbi.nlm.nih.gov/pmc/articles/PMC2797101}{Managing
{Addiction} as a {Chronic Condition}}. \emph{Addiction Science \&
Clinical Practice}, \emph{4}(1), 45--55.

\bibitem[\citeproctext]{ref-derogatislBriefSymptomInventory}
Derogatis, L.R. (2000). \emph{Brief {Symptom Inventory} 18 -
{Administration}, scoring, and procedures manual}. NCS Pearson.

\bibitem[\citeproctext]{ref-flanneryPsychometricPropertiesPenn1999}
Flannery, B. A., Volpicelli, J. R., \& Pettinati, H. M. (1999).
\href{https://www.ncbi.nlm.nih.gov/pubmed/10470970}{Psychometric
properties of the {Penn Alcohol Craving Scale}}. \emph{Alcoholism,
Clinical and Experimental Research}, \emph{23}(8), 1289--1295.

\bibitem[\citeproctext]{ref-hurlbutAssessingAlcoholProblems1992}
Hurlbut, S. C., \& Sher, K. J. (1992). Assessing alcohol problems in
college students. \emph{Journal of American College Health.},
\emph{41}(2), 49--58.

\bibitem[\citeproctext]{ref-marlattRelapsePreventionSecond2007}
Marlatt, G. A., \& Donovan, D. M. (Eds.). (2007). \emph{Relapse
{Prevention}, {Second Edition}: {Maintenance Strategies} in the
{Treatment} of {Addictive Behaviors}} (2nd edition). The Guilford Press.

\bibitem[\citeproctext]{ref-marlattRelapsePreventionMaintenance1985}
Marlatt, G. A., \& Gordon, J. R. (Eds.). (1985). \emph{Relapse
{Prevention}: {Maintenance Strategies} in the {Treatment} of {Addictive
Behaviors}} (First edition). The Guilford Press.

\bibitem[\citeproctext]{ref-mcdowellPreferencesAttitudesDigital2025}
McDowell, B., Dumais, K. M., Gary, S. T., de Gooijer, I., \& Ward, T.
(2025). Preferences and {Attitudes Towards Digital Communication} and
{Symptom Reporting Methods} in {Clinical Trials}. \emph{Patient
Preference and Adherence}, \emph{19}, 255--263.
\url{https://doi.org/10.2147/PPA.S474535}

\bibitem[\citeproctext]{ref-mckiernanDevelopmentBriefAbstinence2011}
McKiernan, P., Cloud, R., Patterson, D. A., Wolf, S., Golder, S., \&
Besel, K. (2011). Development of a {Brief Abstinence Self-Efficacy
Measure}. \emph{Journal of Social Work Practice in the Addictions},
\emph{11}(3), 245--253.
\url{https://doi.org/10.1080/1533256X.2011.593445}

\bibitem[\citeproctext]{ref-mclellanDrugDependenceChronic2000}
McLellan, A. T., Lewis, D. C., O'Brien, C. P., \& Kleber, H. D. (2000).
Drug dependence, a chronic medical illness: Implications for treatment,
insurance, and outcomes evaluation. \emph{JAMA}, \emph{284}(13),
1689--1695. \url{https://doi.org/10.1001/jama.284.13.1689}

\bibitem[\citeproctext]{ref-mohrPersonalSensingUnderstanding2017}
Mohr, D. C., Zhang, M., \& Schueller, S. M. (2017). Personal {Sensing}:
{Understanding Mental Health Using Ubiquitous Sensors} and {Machine
Learning}. \emph{Annual Review of Clinical Psychology}, \emph{13}(1),
23--47. \url{https://doi.org/10.1146/annurev-clinpsy-032816-044949}

\bibitem[\citeproctext]{ref-rounsavilleLapseRelapseChasing2010}
Rounsaville, D. B. (2010). \emph{Lapse, {Relapse}, and {Chasing} the
{Wagon}: {Post-Treatment Drinking} and {Recovery}} {[}PhD thesis{]}.
University of Maryland, Baltimore County.

\bibitem[\citeproctext]{ref-wyantForecastingRiskAlcoholunderreview}
Wyant, K., Fronk, G. E., Yu, C., Punturieri, C. E., \& Curtin, J. J.
(under review). \emph{Forecasting {Risk} of {Alcohol Lapse} up to {Two
Weeks} in {Advance} using {Time-lagged Machine Learning Models}}.

\bibitem[\citeproctext]{ref-wyantMachineLearningModels2024}
Wyant, K., Sant'Ana, S. J., Fronk, G. E., \& Curtin, J. J. (2024).
Machine learning models for temporally precise lapse prediction in
alcohol use disorder. \emph{Journal of Psychopathology and Clinical
Science}, \emph{133}(7), 527--540.
\url{https://doi.org/10.1037/abn0000901}

\end{CSLReferences}




\end{document}
