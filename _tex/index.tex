% Options for packages loaded elsewhere
\PassOptionsToPackage{unicode}{hyperref}
\PassOptionsToPackage{hyphens}{url}
\PassOptionsToPackage{dvipsnames,svgnames,x11names}{xcolor}
%
\documentclass[
  letterpaper,
  DIV=11,
  numbers=noendperiod]{scrartcl}

\usepackage{amsmath,amssymb}
\usepackage{iftex}
\ifPDFTeX
  \usepackage[T1]{fontenc}
  \usepackage[utf8]{inputenc}
  \usepackage{textcomp} % provide euro and other symbols
\else % if luatex or xetex
  \usepackage{unicode-math}
  \defaultfontfeatures{Scale=MatchLowercase}
  \defaultfontfeatures[\rmfamily]{Ligatures=TeX,Scale=1}
\fi
\usepackage{lmodern}
\ifPDFTeX\else  
    % xetex/luatex font selection
\fi
% Use upquote if available, for straight quotes in verbatim environments
\IfFileExists{upquote.sty}{\usepackage{upquote}}{}
\IfFileExists{microtype.sty}{% use microtype if available
  \usepackage[]{microtype}
  \UseMicrotypeSet[protrusion]{basicmath} % disable protrusion for tt fonts
}{}
\makeatletter
\@ifundefined{KOMAClassName}{% if non-KOMA class
  \IfFileExists{parskip.sty}{%
    \usepackage{parskip}
  }{% else
    \setlength{\parindent}{0pt}
    \setlength{\parskip}{6pt plus 2pt minus 1pt}}
}{% if KOMA class
  \KOMAoptions{parskip=half}}
\makeatother
\usepackage{xcolor}
\setlength{\emergencystretch}{3em} % prevent overfull lines
\setcounter{secnumdepth}{-\maxdimen} % remove section numbering
% Make \paragraph and \subparagraph free-standing
\makeatletter
\ifx\paragraph\undefined\else
  \let\oldparagraph\paragraph
  \renewcommand{\paragraph}{
    \@ifstar
      \xxxParagraphStar
      \xxxParagraphNoStar
  }
  \newcommand{\xxxParagraphStar}[1]{\oldparagraph*{#1}\mbox{}}
  \newcommand{\xxxParagraphNoStar}[1]{\oldparagraph{#1}\mbox{}}
\fi
\ifx\subparagraph\undefined\else
  \let\oldsubparagraph\subparagraph
  \renewcommand{\subparagraph}{
    \@ifstar
      \xxxSubParagraphStar
      \xxxSubParagraphNoStar
  }
  \newcommand{\xxxSubParagraphStar}[1]{\oldsubparagraph*{#1}\mbox{}}
  \newcommand{\xxxSubParagraphNoStar}[1]{\oldsubparagraph{#1}\mbox{}}
\fi
\makeatother


\providecommand{\tightlist}{%
  \setlength{\itemsep}{0pt}\setlength{\parskip}{0pt}}\usepackage{longtable,booktabs,array}
\usepackage{calc} % for calculating minipage widths
% Correct order of tables after \paragraph or \subparagraph
\usepackage{etoolbox}
\makeatletter
\patchcmd\longtable{\par}{\if@noskipsec\mbox{}\fi\par}{}{}
\makeatother
% Allow footnotes in longtable head/foot
\IfFileExists{footnotehyper.sty}{\usepackage{footnotehyper}}{\usepackage{footnote}}
\makesavenoteenv{longtable}
\usepackage{graphicx}
\makeatletter
\def\maxwidth{\ifdim\Gin@nat@width>\linewidth\linewidth\else\Gin@nat@width\fi}
\def\maxheight{\ifdim\Gin@nat@height>\textheight\textheight\else\Gin@nat@height\fi}
\makeatother
% Scale images if necessary, so that they will not overflow the page
% margins by default, and it is still possible to overwrite the defaults
% using explicit options in \includegraphics[width, height, ...]{}
\setkeys{Gin}{width=\maxwidth,height=\maxheight,keepaspectratio}
% Set default figure placement to htbp
\makeatletter
\def\fps@figure{htbp}
\makeatother
% definitions for citeproc citations
\NewDocumentCommand\citeproctext{}{}
\NewDocumentCommand\citeproc{mm}{%
  \begingroup\def\citeproctext{#2}\cite{#1}\endgroup}
\makeatletter
 % allow citations to break across lines
 \let\@cite@ofmt\@firstofone
 % avoid brackets around text for \cite:
 \def\@biblabel#1{}
 \def\@cite#1#2{{#1\if@tempswa , #2\fi}}
\makeatother
\newlength{\cslhangindent}
\setlength{\cslhangindent}{1.5em}
\newlength{\csllabelwidth}
\setlength{\csllabelwidth}{3em}
\newenvironment{CSLReferences}[2] % #1 hanging-indent, #2 entry-spacing
 {\begin{list}{}{%
  \setlength{\itemindent}{0pt}
  \setlength{\leftmargin}{0pt}
  \setlength{\parsep}{0pt}
  % turn on hanging indent if param 1 is 1
  \ifodd #1
   \setlength{\leftmargin}{\cslhangindent}
   \setlength{\itemindent}{-1\cslhangindent}
  \fi
  % set entry spacing
  \setlength{\itemsep}{#2\baselineskip}}}
 {\end{list}}
\usepackage{calc}
\newcommand{\CSLBlock}[1]{\hfill\break\parbox[t]{\linewidth}{\strut\ignorespaces#1\strut}}
\newcommand{\CSLLeftMargin}[1]{\parbox[t]{\csllabelwidth}{\strut#1\strut}}
\newcommand{\CSLRightInline}[1]{\parbox[t]{\linewidth - \csllabelwidth}{\strut#1\strut}}
\newcommand{\CSLIndent}[1]{\hspace{\cslhangindent}#1}

\KOMAoption{captions}{tablesignature}
\makeatletter
\@ifpackageloaded{caption}{}{\usepackage{caption}}
\AtBeginDocument{%
\ifdefined\contentsname
  \renewcommand*\contentsname{Table of contents}
\else
  \newcommand\contentsname{Table of contents}
\fi
\ifdefined\listfigurename
  \renewcommand*\listfigurename{List of Figures}
\else
  \newcommand\listfigurename{List of Figures}
\fi
\ifdefined\listtablename
  \renewcommand*\listtablename{List of Tables}
\else
  \newcommand\listtablename{List of Tables}
\fi
\ifdefined\figurename
  \renewcommand*\figurename{Figure}
\else
  \newcommand\figurename{Figure}
\fi
\ifdefined\tablename
  \renewcommand*\tablename{Table}
\else
  \newcommand\tablename{Table}
\fi
}
\@ifpackageloaded{float}{}{\usepackage{float}}
\floatstyle{ruled}
\@ifundefined{c@chapter}{\newfloat{codelisting}{h}{lop}}{\newfloat{codelisting}{h}{lop}[chapter]}
\floatname{codelisting}{Listing}
\newcommand*\listoflistings{\listof{codelisting}{List of Listings}}
\makeatother
\makeatletter
\makeatother
\makeatletter
\@ifpackageloaded{caption}{}{\usepackage{caption}}
\@ifpackageloaded{subcaption}{}{\usepackage{subcaption}}
\makeatother

\ifLuaTeX
  \usepackage{selnolig}  % disable illegal ligatures
\fi
\usepackage{bookmark}

\IfFileExists{xurl.sty}{\usepackage{xurl}}{} % add URL line breaks if available
\urlstyle{same} % disable monospaced font for URLs
\hypersetup{
  pdftitle={Using Cellular Communication Sensing to Support Early Recovery from Alcohol Use Disorder},
  pdfauthor={Kendra Wyant; Coco Yu; John J. Curtin},
  pdfkeywords={Substance use disorders, Machine learning, Cellular
Sensing},
  colorlinks=true,
  linkcolor={blue},
  filecolor={Maroon},
  citecolor={Blue},
  urlcolor={Blue},
  pdfcreator={LaTeX via pandoc}}


\title{Using Cellular Communication Sensing to Support Early Recovery
from Alcohol Use Disorder}
\author{Kendra Wyant \and Coco Yu \and John J. Curtin}
\date{2025-10-01}

\begin{document}
\maketitle


\section{Introduction}\label{introduction}

One of the biggest challenges in Alcohol Use Disorders (AUD) treatment
stems from the chronic relapsing nature of this disease (Scott et al.,
2005). People can relapse days, weeks, and even years after obtaining
the goal of abstinence. At least 60\% of AUD patients relapse to heavy
drinking within 6 months following treatment
(\textbf{nguyenPredictingRelapseAlcohol2020a?};
\textbf{witkiewitzPredictorsHeavyDrinking2011a?};
\textbf{kirshenbaumQuantitativeReviewUbiquitous2009?}). At most 50\% of
people with an AUD achieve remission after several years
(\textbf{fleuryRemissionSubstanceUse2016?};
\textbf{heymanQuittingDrugsQuantitative2013?}).

Identifying initial lapses in early recovery is critical. Lapses --
single episodes of alcohol use -- are easy to define, have a clear
onset, and are also clinically meaningful. They serve as an early
warning sign of returning back to previous drinking behavior
inconsistent with desired goals (Chung \& Maisto, 2006; Marlatt \&
Donovan, 2005; Witkiewitz \& Marlatt, 2004). Lapse predicts future
lapses, with more frequent ones resulting in increased risks of relapse
(Högström Brandt et al., 1999; Witkiewitz \& Marlatt, 2004).

Current predictions of alcohol lapses rely heavily on self reports,
which can be burdensome to measure in long run. Machine learning models
leveraging ecological momentary assessment (EMA) measures have performed
relatively well to predict goal-inconsistent alcohol use (Wyant et al.,
2024). The surveys were collected up to four times daily for three
months. However, constantly completing surveys makes it burdensome for
AUD patients. Although most EMA relevant mental health research
demonstrated modest compliance rates, their time windows last from two
weeks to three months (Czyz et al., 2018; Hung et al., 2016;
Mackesy-Amiti \& Boodram, 2018; Porras-Segovia et al., 2020; van
Genugten et al., 2020). The study length is insufficient because AUD is
a chronic disease that requires constant risk monitoring. As extended
period of time is anticipated, users' perceived burden of answering
surveys is presumably larger (Mogk et al., 2023). Although minimizing
the number of items in the surveys and the frequency of prompting users
to complete the surveys might help mitigate the associated burden, it
can inevitably reduce the prediction precision and temporal precision of
predictions.

Passive cellular communication sensing represents new opportunities due
to its feasibility, relatively low burden on individuals and continuous
data collection. In a smartphone-based sensing platform the primary
expense on the individual is the smartphone. Smartphone usage is already
widespread. Eighty-five percent of US adults have a smartphone and this
number is consistent across all sociodemographic groups, including those
in recovery programs for substance use
(\textbf{massonHealthrelatedInternetUse2019a?};
\textbf{pewresearchcenterMobileFactSheet2021a?}). Studies collecting
passive data have demonstrated high acceptability from participants and
higher compliance rates compared to active measures (Beukenhorst et al.,
2022; Wyant et al., 2023). Further, risk monitoring using cellular
sensing is temporally sensitive to fluctuating risks. Analyzing
communication patterns can detect potential triggers in time without
actively prompting users to reflect on their feelings at the moment or
report their environment.

Cellular communications, with minimal contextual information, is
embedded with potentially rich information that align with relapse
antecedents. For example, social interactions can have important
influences on drinking behavior
(\textbf{hunter-reelEmphasizingInterpersonalFactors2009a?};
\textbf{alvarezSocialNetworkHeavy2021a?}). We may be able to capture
immediate risk based on who someone is calling or what time of day it
is. Decreased interactions may signify isolation common with depressive
symptoms, reaching out to people in one's social network could signify a
positive coping strategy, or changes in patterns between a single person
in one's social network could indicate conflict (Chih et al., 2014;
Hufford et al., 2003; \textbf{millerHowEffectiveAlcoholism2001a?}).

This study aims at building machine learning models from cellular
communications that identify \emph{who} are at heightened risk for
alcohol lapses, \emph{when} they will lapse, and \emph{why} they are at
increased risk.

\section{Methods}\label{methods}

\subsection{Overview}\label{overview}

This study analyzed data collected from 2017-2019 from a larger grant
funded by National Institute of Alcohol Abuse and Alcoholism (R01
AA024391). In this paper, we focus on methods and measures that are
relevant to this study. Additional details on broader methods and the
full set of measures collected are described elsewhere (see
https://osf.io/w5h9y/ and (Wyant et al., 2023; Wyant et al., 2024)).

\subsection{Participants}\label{participants}

Individuals in early recovery from AUD were recruited from Madison and
surrounding area via social media platforms (e.g., Facebook), referrals
from clinics, and television and radio advertisements. After initial
phone screen, interested individuals came in-person to complete a more
in-depth screening to determine their eligibility. We documented their
demographic information. Inclusion criteria include that participants:
1) must be at least aged 18 or older; 2) must meet criteria for AUD with
at least moderate severity (\textgreater four DSM-5 criteria); 3) must
be abstinent from alcohol for at least one week and fewer than two
months at time of intake; 4) must be able to read and write in English;
5) must be willing to use smartphone and their smartphone is compatible
with our study technology. Participants were excluded if they have a
lifetime history of severe and persistent mental illness. One hundred
sixty-nine participants were eligible and enrolled in the study. After
excluding participants who discontinued before the first follow-up
session and those with low compliance rates and too few communications
(\textless100 messages), we have a final sample size of 150
participants.

\subsection{Procedures}\label{procedures}

The study lasted up to three months with five in-person visits.
Participants completed an in-person screening visit to determine their
eligibility, obtain their informed consent, and collect their
demographic information and self-report measures. They then completed an
intake session one week later and three follow-up visits afterwards
spaced at one-month intervals. During each of the follow-up visits, a
research assistant downloaded participants' SMS messages from their
phone, verified reports of lapses and queried participants about any
additional unreported laspes. Additional self-reported measures were
obtained (see https://osf.io/w5h9y/).

Throughout the course of the study, participants were expected to
complete four daily EMAs that asked about their alcohol cravings, risky
situations, stressful/pleasant events, etc (Wyant et al., 2024).
Notably, in the first item in the EMA survey, participants also reported
their past alcohol use. Answer to this item will be used as the
predicted outcome.

\section{Results}\label{results}

\section{Discussion}\label{discussion}

\newpage

\phantomsection\label{refs}
\begin{CSLReferences}{1}{0}
\bibitem[\citeproctext]{ref-beukenhorstUsingSmartphonesReduce2022a}
Beukenhorst, A. L., Burke, K. M., Scheier, Z., Miller, T. M., Paganoni,
S., Keegan, M., Collins, E., Connaghan, K. P., Tay, A., Chan, J., Berry,
J. D., \& Onnela, J.-P. (2022). Using smartphones to reduce research
burden in a neurodegenerative population and assessing participant
adherence: {A} randomized clinical trial and two observational studies.
\emph{JMIR Mhealth and Uhealth}, \emph{10}(2), e31877.
\url{https://doi.org/10.2196/31877}

\bibitem[\citeproctext]{ref-chihPredictiveModelingAddiction2014a}
Chih, M.-Y., Patton, T., McTavish, F. M., Isham, A. J., Judkins-Fisher,
C. L., Atwood, A. K., \& Gustafson, D. H. (2014). Predictive modeling of
addiction lapses in a mobile health application. \emph{Journal of
Substance Abuse Treatment}, \emph{46}(1), 29--35.
\url{https://doi.org/10.1016/j.jsat.2013.08.004}

\bibitem[\citeproctext]{ref-chungRelapseAlcoholOther2006a}
Chung, T., \& Maisto, S. A. (2006). Relapse to alcohol and other drug
use in treated adolescents: {Review} and reconsideration of relapse as a
change point in clinical course. \emph{Clinical Psychology Review},
\emph{26}(2), 149--161. \url{https://doi.org/10.1016/j.cpr.2005.11.004}

\bibitem[\citeproctext]{ref-czyzEcologicalAssessmentDaily2018}
Czyz, E. K., King, C. A., \& Nahum-Shani, I. (2018). Ecological
assessment of daily suicidal thoughts and attempts among suicidal teens
after psychiatric hospitalization: {Lessons} about feasibility and
acceptability. \emph{Psychiatry Research}, \emph{267}, 566--574.
\url{https://doi.org/10.1016/j.psychres.2018.06.031}

\bibitem[\citeproctext]{ref-hogstrombrandtPredictionSingleEpisodes1999a}
Högström Brandt, A. M., Thorburn, D., Hiltunen, A. J., \& Borg, S.
(1999). Prediction of single episodes of drinking during the treatment
of alcohol-dependent patients. \emph{Alcohol (Fayetteville, N.Y.)},
\emph{18}(1), 35--42.
\url{https://doi.org/10.1016/s0741-8329(98)00065-2}

\bibitem[\citeproctext]{ref-huffordRelapseNonlinearDynamic2003a}
Hufford, M. R., Witkiewitz, K., Shields, A. L., Kodya, S., \& Caruso, J.
C. (2003). Relapse as a nonlinear dynamic system: {Application} to
patients with alcohol use disorders. \emph{Journal of Abnormal
Psychology}, \emph{112}(2), 219--227.
\url{https://doi.org/10.1037/0021-843X.112.2.219}

\bibitem[\citeproctext]{ref-hungSmartphonebasedEcologicalMomentary2016}
Hung, S., Li, M.-S., Chen, Y.-L., Chiang, J.-H., Chen, Y.-Y., \& Hung,
G. C.-L. (2016). Smartphone-based ecological momentary assessment for
{Chinese} patients with depression: {An} exploratory study in {Taiwan}.
\emph{Asian Journal of Psychiatry}, \emph{23}, 131--136.
\url{https://doi.org/10.1016/j.ajp.2016.08.003}

\bibitem[\citeproctext]{ref-mackesy-amitiFeasibilityEcologicalMomentary2018}
Mackesy-Amiti, M. E., \& Boodram, B. (2018). Feasibility of ecological
momentary assessment to study mood and risk behavior among young people
who inject drugs. \emph{Drug and Alcohol Dependence}, \emph{187},
227--235. \url{https://doi.org/10.1016/j.drugalcdep.2018.03.016}

\bibitem[\citeproctext]{ref-marlattRelapsePreventionMaintenance2005a}
Marlatt, G. A., \& Donovan, D. M. (Eds.). (2005). \emph{Relapse
prevention: {Maintenance} strategies in the treatment of addictive
behaviors, 2nd ed} (pp. xiv, 416). The Guilford Press.

\bibitem[\citeproctext]{ref-mogkImplementationWorkflowStrategies2023}
Mogk, J. M., Matson, T. E., Caldeiro, R. M., Garza Mcwethy, A. M.,
Beatty, T., Sevey, B. C., Hsu, C. W., \& Glass, J. E. (2023).
Implementation and workflow strategies for integrating digital
therapeutics for alcohol use disorders into primary care: {A}
qualitative study. \emph{Addiction Science \& Clinical Practice},
\emph{18}(1). \url{https://doi.org/10.1186/s13722-023-00387-w}

\bibitem[\citeproctext]{ref-porras-segoviaSmartphonebasedEcologicalMomentary2020}
Porras-Segovia, A., Molina-Madueño, R. M., Berrouiguet, S.,
López-Castroman, J., Barrigón, M. L., Pérez-Rodríguez, M. S., Marco, J.
H., Díaz-Oliván, I., de León, S., Courtet, P., Artés-Rodríguez, A., \&
Baca-García, E. (2020). Smartphone-based ecological momentary assessment
({EMA}) in psychiatric patients and student controls: {A} real-world
feasibility study. \emph{Journal of Affective Disorders}, \emph{274},
733--741. \url{https://doi.org/10.1016/j.jad.2020.05.067}

\bibitem[\citeproctext]{ref-scottPathwaysRelapseTreatment2005}
Scott, C. K., Foss, M. A., \& Dennis, M. L. (2005). Pathways in the
relapse--treatment--recovery cycle over 3 years. \emph{Journal of
Substance Abuse Treatment}, \emph{28 Suppl 1}, S63--72.
\url{https://doi.org/10.1016/j.jsat.2004.09.006}

\bibitem[\citeproctext]{ref-vangenugtenExperiencedBurdenAdherence2020a}
van Genugten, C. R., Schuurmans, J., Lamers, F., Riese, H., Penninx, B.
W. J. H., Schoevers, R. A., Riper, H. M., \& Smit, J. H. (2020).
Experienced {Burden} of and {Adherence} to {Smartphone-Based Ecological
Momentary Assessment} in {Persons} with {Affective Disorders}.
\emph{Journal of Clinical Medicine}, \emph{9}(2), 322.
\url{https://doi.org/10.3390/jcm9020322}

\bibitem[\citeproctext]{ref-witkiewitzRelapsePreventionAlcohol2004b}
Witkiewitz, K., \& Marlatt, G. A. (2004). Relapse prevention for alcohol
and drug problems: That was {Zen}, this is {Tao}. \emph{The American
Psychologist}, \emph{59}(4), 224--235.
\url{https://doi.org/10.1037/0003-066X.59.4.224}

\bibitem[\citeproctext]{ref-wyantAcceptabilityPersonalSensing2023a}
Wyant, K., Moshontz, H., Ward, S. B., Fronk, G. E., \& Curtin, J. J.
(2023). Acceptability of {Personal Sensing Among People With Alcohol Use
Disorder}: {Observational Study}. \emph{JMIR mHealth and uHealth},
\emph{11}, e41833. \url{https://doi.org/10.2196/41833}

\bibitem[\citeproctext]{ref-wyantMachineLearningModels2024}
Wyant, K., Sant'Ana, S. J., Fronk, G. E., \& Curtin, J. J. (2024).
Machine learning models for temporally precise lapse prediction in
alcohol use disorder. \emph{Journal of Psychopathology and Clinical
Science}, \emph{133}(7), 527--540.
\url{https://doi.org/10.1037/abn0000901}

\end{CSLReferences}




\end{document}
