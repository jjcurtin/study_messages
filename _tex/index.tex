% Options for packages loaded elsewhere
\PassOptionsToPackage{unicode}{hyperref}
\PassOptionsToPackage{hyphens}{url}
\PassOptionsToPackage{dvipsnames,svgnames,x11names}{xcolor}
%
\documentclass[
  letterpaper,
  DIV=11,
  numbers=noendperiod]{scrartcl}

\usepackage{amsmath,amssymb}
\usepackage{iftex}
\ifPDFTeX
  \usepackage[T1]{fontenc}
  \usepackage[utf8]{inputenc}
  \usepackage{textcomp} % provide euro and other symbols
\else % if luatex or xetex
  \usepackage{unicode-math}
  \defaultfontfeatures{Scale=MatchLowercase}
  \defaultfontfeatures[\rmfamily]{Ligatures=TeX,Scale=1}
\fi
\usepackage{lmodern}
\ifPDFTeX\else  
    % xetex/luatex font selection
\fi
% Use upquote if available, for straight quotes in verbatim environments
\IfFileExists{upquote.sty}{\usepackage{upquote}}{}
\IfFileExists{microtype.sty}{% use microtype if available
  \usepackage[]{microtype}
  \UseMicrotypeSet[protrusion]{basicmath} % disable protrusion for tt fonts
}{}
\makeatletter
\@ifundefined{KOMAClassName}{% if non-KOMA class
  \IfFileExists{parskip.sty}{%
    \usepackage{parskip}
  }{% else
    \setlength{\parindent}{0pt}
    \setlength{\parskip}{6pt plus 2pt minus 1pt}}
}{% if KOMA class
  \KOMAoptions{parskip=half}}
\makeatother
\usepackage{xcolor}
\setlength{\emergencystretch}{3em} % prevent overfull lines
\setcounter{secnumdepth}{-\maxdimen} % remove section numbering
% Make \paragraph and \subparagraph free-standing
\makeatletter
\ifx\paragraph\undefined\else
  \let\oldparagraph\paragraph
  \renewcommand{\paragraph}{
    \@ifstar
      \xxxParagraphStar
      \xxxParagraphNoStar
  }
  \newcommand{\xxxParagraphStar}[1]{\oldparagraph*{#1}\mbox{}}
  \newcommand{\xxxParagraphNoStar}[1]{\oldparagraph{#1}\mbox{}}
\fi
\ifx\subparagraph\undefined\else
  \let\oldsubparagraph\subparagraph
  \renewcommand{\subparagraph}{
    \@ifstar
      \xxxSubParagraphStar
      \xxxSubParagraphNoStar
  }
  \newcommand{\xxxSubParagraphStar}[1]{\oldsubparagraph*{#1}\mbox{}}
  \newcommand{\xxxSubParagraphNoStar}[1]{\oldsubparagraph{#1}\mbox{}}
\fi
\makeatother


\providecommand{\tightlist}{%
  \setlength{\itemsep}{0pt}\setlength{\parskip}{0pt}}\usepackage{longtable,booktabs,array}
\usepackage{calc} % for calculating minipage widths
% Correct order of tables after \paragraph or \subparagraph
\usepackage{etoolbox}
\makeatletter
\patchcmd\longtable{\par}{\if@noskipsec\mbox{}\fi\par}{}{}
\makeatother
% Allow footnotes in longtable head/foot
\IfFileExists{footnotehyper.sty}{\usepackage{footnotehyper}}{\usepackage{footnote}}
\makesavenoteenv{longtable}
\usepackage{graphicx}
\makeatletter
\newsavebox\pandoc@box
\newcommand*\pandocbounded[1]{% scales image to fit in text height/width
  \sbox\pandoc@box{#1}%
  \Gscale@div\@tempa{\textheight}{\dimexpr\ht\pandoc@box+\dp\pandoc@box\relax}%
  \Gscale@div\@tempb{\linewidth}{\wd\pandoc@box}%
  \ifdim\@tempb\p@<\@tempa\p@\let\@tempa\@tempb\fi% select the smaller of both
  \ifdim\@tempa\p@<\p@\scalebox{\@tempa}{\usebox\pandoc@box}%
  \else\usebox{\pandoc@box}%
  \fi%
}
% Set default figure placement to htbp
\def\fps@figure{htbp}
\makeatother
% definitions for citeproc citations
\NewDocumentCommand\citeproctext{}{}
\NewDocumentCommand\citeproc{mm}{%
  \begingroup\def\citeproctext{#2}\cite{#1}\endgroup}
\makeatletter
 % allow citations to break across lines
 \let\@cite@ofmt\@firstofone
 % avoid brackets around text for \cite:
 \def\@biblabel#1{}
 \def\@cite#1#2{{#1\if@tempswa , #2\fi}}
\makeatother
\newlength{\cslhangindent}
\setlength{\cslhangindent}{1.5em}
\newlength{\csllabelwidth}
\setlength{\csllabelwidth}{3em}
\newenvironment{CSLReferences}[2] % #1 hanging-indent, #2 entry-spacing
 {\begin{list}{}{%
  \setlength{\itemindent}{0pt}
  \setlength{\leftmargin}{0pt}
  \setlength{\parsep}{0pt}
  % turn on hanging indent if param 1 is 1
  \ifodd #1
   \setlength{\leftmargin}{\cslhangindent}
   \setlength{\itemindent}{-1\cslhangindent}
  \fi
  % set entry spacing
  \setlength{\itemsep}{#2\baselineskip}}}
 {\end{list}}
\usepackage{calc}
\newcommand{\CSLBlock}[1]{\hfill\break\parbox[t]{\linewidth}{\strut\ignorespaces#1\strut}}
\newcommand{\CSLLeftMargin}[1]{\parbox[t]{\csllabelwidth}{\strut#1\strut}}
\newcommand{\CSLRightInline}[1]{\parbox[t]{\linewidth - \csllabelwidth}{\strut#1\strut}}
\newcommand{\CSLIndent}[1]{\hspace{\cslhangindent}#1}

\KOMAoption{captions}{tablesignature}
\makeatletter
\@ifpackageloaded{caption}{}{\usepackage{caption}}
\AtBeginDocument{%
\ifdefined\contentsname
  \renewcommand*\contentsname{Table of contents}
\else
  \newcommand\contentsname{Table of contents}
\fi
\ifdefined\listfigurename
  \renewcommand*\listfigurename{List of Figures}
\else
  \newcommand\listfigurename{List of Figures}
\fi
\ifdefined\listtablename
  \renewcommand*\listtablename{List of Tables}
\else
  \newcommand\listtablename{List of Tables}
\fi
\ifdefined\figurename
  \renewcommand*\figurename{Figure}
\else
  \newcommand\figurename{Figure}
\fi
\ifdefined\tablename
  \renewcommand*\tablename{Table}
\else
  \newcommand\tablename{Table}
\fi
}
\@ifpackageloaded{float}{}{\usepackage{float}}
\floatstyle{ruled}
\@ifundefined{c@chapter}{\newfloat{codelisting}{h}{lop}}{\newfloat{codelisting}{h}{lop}[chapter]}
\floatname{codelisting}{Listing}
\newcommand*\listoflistings{\listof{codelisting}{List of Listings}}
\makeatother
\makeatletter
\makeatother
\makeatletter
\@ifpackageloaded{caption}{}{\usepackage{caption}}
\@ifpackageloaded{subcaption}{}{\usepackage{subcaption}}
\makeatother

\usepackage{bookmark}

\IfFileExists{xurl.sty}{\usepackage{xurl}}{} % add URL line breaks if available
\urlstyle{same} % disable monospaced font for URLs
\hypersetup{
  pdftitle={Evaluating Cellular Communication Sensing for Lapse Risk Prediction During Early Recovery from Alcohol Use Disorder},
  pdfauthor={Kendra Wyant; Jiachen Yu; John J. Curtin},
  pdfkeywords={Substance use disorders, Precision Medicine, Machine
learning, Digital Phenotyping, Relapse Prevention},
  colorlinks=true,
  linkcolor={blue},
  filecolor={Maroon},
  citecolor={Blue},
  urlcolor={Blue},
  pdfcreator={LaTeX via pandoc}}


\title{Evaluating Cellular Communication Sensing for Lapse Risk
Prediction During Early Recovery from Alcohol Use Disorder}
\author{Kendra Wyant \and Jiachen Yu \and John J. Curtin}
\date{2026-01-23}

\begin{document}
\maketitle
\begin{abstract}
Alcohol Use Disorder (AUD) is a chronic, relapsing disease. An automated
recovery support system using personal sensing and machine learning may
help identify when individuals are at elevated lapse risk. Cellular
communication sensing may detect dynamic changes in lapse risk and can
be contextualized with self-reported, risk-relevant information about
contacts. We evaluated a machine learning model predicting next-day
alcohol lapse among individuals in early recovery from AUD using
contextualized cellular communication data and baseline demographic and
AUD characteristics. A total of 144 participants (49\% male; mean
age=40; 87\% non-Hispanic White) with a goal of abstinence provided
cellular communication data and alcohol use reports via a 4x daily EMA
for up to three months. Models were trained and evaluated using repeated
k-fold cross-validation. The best-performing model used an elastic net
algorithm and retained 13 features (median posterior auROC=0.68, 95\%
Bayesian credible interval (CI; {[}0.64, 0.71{]}). A baseline comparison
model including only baseline features retained five features and
demonstrated nearly identical performance (median auROC=0.68, 95\% CI
{[}0.64, 0.71{]}). Cellular communication data capture some
risk-relevant signal for alcohol lapse but do not provide incremental
predictive value beyond baseline measures. Several communication
features were retained in the final model with moderately sized
coefficients, suggesting that aspects of social communication may be
important for understanding lapse risk. Although, limitations inherent
to cellular communication as a sensing method may outweigh their added
value.
\end{abstract}


\textsubscript{Source:
\href{https://jjcurtin.github.io/study_messages/index.qmd.html}{Article
Notebook}}

\section{Introduction}\label{introduction}

Alcohol Use Disorder (AUD) is a chronic, relapsing disease {[}1--3{]}.
Lapses, single episodes of alcohol use, are among the strongest
predictors (and a necessary precursor) for relapse, a full return to
harmful drinking {[}4,5{]}. While lapses can occur at any point in
recovery, they are particularly risky during early recovery {[}6{]}.
Protective coping mechanisms and socio-environmental resources that
support recovery are dynamic and accumulate over time {[}7{]}. As a
result, early recovery represents a critical window of vulnerability
during which a lapse is more likely to escalate into relapse.

An automated recovery support system powered by personal sensing and
machine learning may assist with the inherently difficult task of
identifying when and why someone is at increased risk for lapse.
Personal sensing of densely sampled data from individuals' day-to-day
lives can provide the inputs necessary for temporally dynamic lapse
predictions {[}8{]}. Early machine learning models using ecological
momentary assessment (EMA) data have achieved excellent accuracy in
predicting future lapses back to alcohol use in treatment seeking
populations {[}9--11{]}.

Despite the high predictive success of EMA, questions remain about the
long-term feasibility of a self-report sensing method. EMA has been
shown to be well-tolerated among substance use populations over
relatively short periods of time {[}12,13{]}. It is unclear whether
individuals would be willing and able to adhere to an extensive EMA
protocol (e.g., 4 prompts per day) indefinitely. Moreover, EMA items are
chosen using domain expertise from decades of research on the
self-report factors that predict lapse. It is possible, however, that
there are several alternative precipitators of lapse not yet discovered
due to small subtle changes in one's environment, social circle, or
lifestyle that cannot be easily identified via self-report.

Cellular communication sensing may be a promising alternative to EMA.
Whereas EMA is limited to, at most, several assessments per day,
communication sensing is mostly passive and can be monitored
moment-by-moment. Cellular communication patterns capture clear,
risk-relevant constructs. Late-night phone calls could indicate an
emergency, ``drunk dialing,'' or interpersonal conflict. A decrease in
the number of contacts an individual communicates with could reflect a
shrinking social circle, isolation, or disengagement. Furthermore,
cellular communication sensing enables data-driven feature engineering,
whereby features are systematically derived from raw communication logs
and retained based on their predictive utility rather than a priori
theoretical assumptions.

These data may become even more powerful when communication patterns are
contextualized with participant-specific meaning. For instance, knowing
a participant's relationship to their contacts, whether they have
previously drunk alcohol with a given contact, or whether that contact
supports their recovery goals could alter interpretation. In the
examples above, contextualized communication data might reveal that the
late-night calls are made to a sponsor, or that a shrinking social
circle reflects reduced contact with individuals unsupportive of their
recovery. In this way, the same communication patterns may reflect
protective processes rather than increased lapse risk.

In this study, we assessed whether contextualized cellular communication
features contain clinically meaningful signals for predicting next-day
alcohol lapse risk among individuals in early recovery from AUD. Using a
machine learning model, we evaluated the predictive utility of these
features and identified the most important communication features, with
the goal of uncovering new, clinically meaningful predictors of lapse
risk.

\section{Methods}\label{methods}

\subsection{Transparency and Openness}\label{transparency-and-openness}

We adhere to research transparency principles that are crucial for
robust and replicable science. First, we reported how we determined the
sample size, all data exclusions, all manipulations, and all study
measures. We provide a transparency checklist {[}14{]} in the
supplement. Second, our features, labels, questionnaires, and other
study materials are publicly available on our Open Science Framework
(OSF) page (\url{https://osf.io/wgpz9/}). Finally, the annotated
analysis scripts are publicly available on our study website
(\url{https://jjcurtin.github.io/study_messages/}).

\subsection{Participants and
Procedure}\label{participants-and-procedure}

We recruited 192 adults in early recovery from AUD in Madison,
Wisconsin, USA through print and digital advertisements and partnerships
with treatment centers from February 15, 2017 through September 19,
2019. This sample size was determined based on traditional power
analysis methods for logistic regression {[}15{]} because comparable
approaches for machine learning models have not yet been validated.
Eligibility criteria required that participants were age 18 or older,
able to read and write in English, had moderate to severe AUD (≥4
self-reported DSM-5 symptoms), had a goal of abstinence from alcohol at
the time of the screening visit, had been abstinent for 1--8 weeks, were
willing to use a single smartphone, and were not exhibiting severe
psychosis or paranoia (defined as scores \textgreater2.2 or 2.8,
respectively, on the psychosis or paranoia scales of the Symptom
Checklist--90 {[}16{]}.)

Participants completed up to 5 study visits over approximately 3 months:
a screening visit, intake visit, and 3 monthly follow-up visits. At
screening we determined eligibility and collected demographic
information (age, sex at birth, race, ethnicity, education, marital
status, employment, and income) and clinical characteristics (DSM-5 AUD
symptom count, alcohol problems {[}17{]}, and presence of psychological
symptoms {[}16{]}). At the intake visit, approximately two weeks after
screening, we collected additional self-report data on abstinence
self-efficacy {[}18{]}, craving {[}19{]}, and recent recovery efforts
and goals.

At each monthly follow-up, we downloaded backups of participants'
cellular communication metadata directly from their smartphones.
Metadata included the phone number of the other party, the date and time
of the communication, the origin of call or message (i.e., incoming or
outgoing), whether the call was answered (voice calls only), and the
duration of the call (voice calls only). During each follow-up visit,
study staff identified important contacts. Contacts that participants
communicated with at least twice by call or text in the past month were
considered important. For each important contact, participants answered
seven contextual questions about their type of relationship, whether
they ever drank alcohol with this person, the drinking status of the
contact, expectations about whether the contact would drink in their
presence, recovery status of contact, level of supportiveness of
contact, and affective experiences with the contact.

While enrolled, participants completed four brief daily ecological
momentary assessments (7-10 questions). The first item assessed alcohol
use (date and time of any unreported drinking episodes). The remaining
EMA questions were used as features in other studies {[}10,11{]}, but
were outside the scope of this cellular communication sensing study.
Additional sensing data streams and self-report measures were collected
for the parent grant. We compensated participants up to \$115 per month
for completing study tasks (i.e., EMAs, monthly follow-up visits and
sharing sensing data) and \$66 per month to offset the cost of their
cellphone plan. The full study protocol is available on our OSF page
(\url{https://osf.io/wgpz9/}).

\subsection{Ethics}\label{ethics}

All procedures were approved by the University of Wisconsin-Madison
Institutional Review Board (Study \#2015-0780) and carried out in
accordance with the principles of the Declaration of Helsinki. All
participants provided written informed consent observed by a research
assistant.

\subsection{Data Analysis Plan}\label{data-analysis-plan}

\subsubsection{Labels}\label{labels}

Our models predicted the probability of an alcohol lapse within a
24-hour window. Predictions were generated daily at 4 a.m., beginning on
participants' second study day and continuing for up to 3 months.
Participants reported the date and hour of the start and end time of any
alcohol use on the first item of the EMA. Prediction windows were
labeled as lapse if any alcohol use was reported in the 24-hour window.
In total, there were 11,507 labeled prediction windows across all
participants. Positive lapse labels were underrepresented (7.5\%;
861/11,507).

\subsubsection{Feature Engineering}\label{feature-engineering}

We filtered the raw comunication data to include only communications
with known context (i.e., people with whom they communicated with at
least twice in a month and whom they provided self-report context
about). Cellular communication features were engineered from all
available data up to the start of each window. We used six feature
scoring epochs (6, 12, 24, 48, 72, and 168 hours before the start of the
prediction window) to create features.

Within each feature scoring epoch, we calculated two types of features:
raw and difference features. Raw features represent the feature value
calculated within a given feature scoring epoch. For example, the raw
rate of incoming text messages in a 48-hour feature scoring epoch was
calculated as the total number of incoming text messages in the 48 hours
immediately preceding the start of the prediction window, divided by 48.
Difference features capture participant-level changes from baseline.
Specifically, for each feature we subtracted the participant's mean
value (using all available data prior to the prediction window) from the
associated raw feature value. For example, the difference feature for
incoming text messages was calculated as the raw incoming text message
rate minus the participant's average incoming text message rate across
all time on study.

The full model included 406 features from cellular communication data
plus 24 numeric or dummy-coded features from baseline self-report
measures collected at screening and intake visits. We also evaluated a
comparison model that used only the baseline features. Table~\ref{tbl-1}
details the raw predictors, feature engineering procedures, and features
included in the full vs.~baseline models. Other feature engineering
steps performed during cross-validation included imputing missing values
(median imputation for numeric features and mode imputation for nominal
features), standardizing all features, and removing zero and near-zero
variance features as determined from held-in data.

\begin{longtable}[]{@{}llllrll@{}}

\toprule\noalign{}
Raw Predictor & Response Options & Feature Engineering & Feature Scoring
Epochs & Total Features & Full Model & Baseline Model \\
\midrule\noalign{}
\endhead
\midrule\noalign{}
{Note: } & & & & & & \\
\textsuperscript{} Cellular communication features were scored over six
feature scoring epochs (6, 12, 24, 48, 72, and 168 hours before the
start of the prediction window). Within each feature scoring epoch we
calculated two types of features: raw and difference features. Raw
features represent the raw feature value calculated within a scoring
epoch (e.g., the rate count of text messages during the 48 hours
immediately preceding the start of the prediction window). Difference
features capture participant-level changes from their baseline scores
(e.g., the participant's average rate count of text messages across all
time on study subtracted from the rate count in the preceding 48 hours).
& & & & & & \\
\bottomrule\noalign{}
\endlastfoot
Originated & Incoming, outgoing & Difference and raw rate counts for
text messages and voice calls & 6, 12, 24, 48, 72, and 168 hours & 48 &
Yes & No \\
Call duration & Duration (in minutes) & Difference and raw rate sums of
duration, difference and raw most recent duration & 6, 12, 24, 48, 72,
and 168 hours & 14 & Yes & No \\
Call answered & Yes, no & Difference and raw rate counts for unanswered
incoming voice calls & 6, 12, 24, 48, 72, and 168 hours & 12 & Yes &
No \\
Date/time of communication & Date and time & Difference and raw rate
counts for text messages and voice calls at night (10 pm -- 6am) and on
weekends & 24, 48, 72, and 168 hours (night), 168 hours (weekend) & 20 &
Yes & No \\
Phone number & Phone number & Difference and raw rate counts of unique
phone numbers & 6, 12, 24, 48, 72, and 168 hours & 12 & Yes & No \\
Type of Relationship & Family, friend, counselor or social worker,
co-worker & Difference and raw rate counts of unique phone numbers & 6,
12, 24, 48, 72, and 168 hours & 48 & Yes & No \\
Have you drank alcohol with this person? & Never/almost never,
occasionally, almost always/always & Difference and raw rate counts of
each response option & 6, 12, 24, 48, 72, and 168 hours & 36 & Yes &
No \\
What is their drinking status? & Drinker, non-drinker, don't know &
Difference and raw rate counts of each response option & 6, 12, 24, 48,
72, and 168 hours & 36 & Yes & No \\
Would you expect them to drink in your presence? & Yes, no, uncertain &
Difference and raw rate counts of each response option & 6, 12, 24, 48,
72, and 168 hours & 36 & Yes & No \\
Are they currently in recovery from drugs or alcohol? & Yes, no, don't
know & Difference and raw rate counts of each response option & 6, 12,
24, 48, 72, and 168 hours & 36 & Yes & No \\
Are they supportive about your recovery goals? & Supportive,
unsupportive, mixed, neutral, don't know & Difference and raw rate
counts of each response option & 6, 12, 24, 48, 72, and 168 hours & 60 &
Yes & No \\
How are your typical experiences with this person? & Pleasant,
unpleasant, mixed, neutral & Difference and raw rate counts of each
response option & 6, 12, 24, 48, 72, and 168 hours & 48 & Yes & No \\
DSM-5 symptom count & Numeric (4-11) & & & 1 & Yes & Yes \\
Past year alcohol problems & Numeric (0-27) & & & 1 & Yes & Yes \\
Craving & Numeric (0-30) & & & 1 & Yes & Yes \\
Abstinence self-efficacy: Negative affect, social, physical, and craving
subscales & Numeric (0-20) & & & 4 & Yes & Yes \\
Number of individual alcohol counseling sessions attended (past 30 days)
& Numeric & & & 1 & Yes & Yes \\
Number of group alcohol counseling sessions attended (past 30 days) &
Numeric & & & 1 & Yes & Yes \\
Number of self-help group meetings attended (past 30 days) & Numeric & &
& 1 & Yes & Yes \\
Number of other mental health counseling sessions attended (past 30
days) & Numeric & & & 1 & Yes & Yes \\
Number of days in contact with supportive people (past 30 days) &
Numeric & & & 1 & Yes & Yes \\
Number of days in contact with unsupportive people (past 30 days) &
Numeric & & & 1 & Yes & Yes \\
Taken prescribed medication for alcohol use disorder (past 30 days) &
Yes, no & Dummy coded & & 1 & Yes & Yes \\
Taken prescribed medication for other mental health disorder (past 30
days) & Yes, no & Dummy coded & & 1 & Yes & Yes \\
Satisfaction with progress toward recovery goals (past 30 days) &
Numeric (0-4) & & & 1 & Yes & Yes \\
Confidence in abstinence ability (next 30 days) & Numeric (0-4) & & & 1
& Yes & Yes \\
Has a goal of abstinence & Yes, no, uncertain & Dummy coded & & 2 & Yes
& Yes \\
Age & Numeric (years) & & & 1 & Yes & Yes \\
Sex at birth & Male, female & Dummy coded & & 1 & Yes & Yes \\
Race & Non-Hispanic White, non-White and/or Hispanic & Dummy coded & & 1
& Yes & Yes \\
Education & High school or less, some college, college degree & Dummy
coded & & 2 & Yes & Yes \\
Income & Numeric (dollars) & & & 1 & Yes & Yes \\
Marital Status & Married, not married, other & Dummy coded & & 2 & Yes &
Yes \\


\caption{\label{tbl-1}Raw Predictors, Response Options, Feature
Engineering Methods, Feature Scoring Epochs, Total Number of Features,
and Indication of Inclusion in Full and Baseline Models}

\tabularnewline
\end{longtable}

\textsubscript{Source:
\href{https://jjcurtin.github.io/study_messages/notebooks/mak_tables-preview.html\#cell-tbl-1}{Make
All Tables for Main Manuscript}}

\subsubsection{Model Selection and
Evaluation}\label{model-selection-and-evaluation}

Candidate model configurations differed by algorithm (elastic net,
random forest, XGBoost), outcome resampling method (i.e., up-sampling
and down-sampling of the outcome at ratios ranging from 5:1 to 1:1), and
hyperparameter values. The best configuration for each model was
selected using 6 repeats of 5-fold cross-validation. Participants were
grouped so that all of their data were always in the held-in or held-out
fold for a split, but never in both. Our performance metric was area
under the receiver operating curve (auROC). Folds were stratified so
that all folds contained comparable proportions of individuals who
lapsed frequently (i.e., 10+ times).

We evaluated model performance with a Bayesian hierarchical generalized
linear model. Posterior distributions with 95\% credible intervals (CI)
were estimated from the 30 held-out folds using weakly informative,
data-dependent priors to regularize and reduce overfitting. (Residual SD
\textasciitilde{} normal(0, exp(2)); intercept (centered predictors)
\textasciitilde{} normal(2.3, 1.3); window-width contrasts
\textasciitilde{} normal(0, 2.69); covariance \textasciitilde{}
decov(1,1,1,1)). Random intercepts were included for repeat and fold
(nested within repeat). auROCs were logit-transformed and regressed on
model type to estimate the probability that model performances differed
systematically.

The best performing model used an elastic net algorithm. This model was
refit on the entire data set to identify the most important features. We
quantified feature importance by examining the retained features (i.e.,
coefficient value \textgreater{} 0) in the full model and ordering them
by absolute coefficient value. These values provide an estimate of the
direction and magnitude of association between each predictor and the
outcome, conditional on the other features retained.

\section{Results}\label{results}

\subsection{Participants}\label{participants}

We screened 192 participants. Of these, 169 enrolled during the intake
visit and 154 completed the first follow-up visit. We excluded data from
one participant due to drinking multiple times a day every day on study,
suggesting they did not have a goal of abstinence. We excluded data from
one participant due to evidence of careless responding. We excluded data
from one participant due to poor compliance with EMA resulting in
questionable lapse labels. We excluded data from seven participants due
to poor compliance providing communication data (i.e., deleting logs
prior to the download or not providing context information about
important contacts). The final analytic sample included 144
participants. Table~\ref{tbl-2} provides the demographic
characterization of our sample. 56\% of participants reported at least
one lapse while on study.

\begin{longtable}[]{@{}lrrlll@{}}

\toprule\noalign{}
& N & \% & M & SD & Range \\
\midrule\noalign{}
\endhead
\midrule\noalign{}
{Note: } & & & & & \\
\textsuperscript{} N = 144 & & & & & \\
\bottomrule\noalign{}
\endlastfoot
Age & & & 40.4 & 11.8 & 21-72 \\
Sex at Birth & & & & & \\
\multicolumn{6}{@{}l@{}}{%
\textbf{}} \\
Female & 74 & 51.4 & & & \\
Male & 70 & 48.6 & & & \\
Race & & & & & \\
\multicolumn{6}{@{}l@{}}{%
\textbf{}} \\
American Indian/Alaska Native & 3 & 2.1 & & & \\
Asian & 2 & 1.4 & & & \\
Black/African American & 8 & 5.6 & & & \\
White/Caucasian & 125 & 86.8 & & & \\
Other/Multiracial & 6 & 4.2 & & & \\
Hispanic, Latino, or Spanish origin & & & & & \\
\multicolumn{6}{@{}l@{}}{%
\textbf{}} \\
Yes & 3 & 2.1 & & & \\
No & 141 & 97.9 & & & \\
Education & & & & & \\
\multicolumn{6}{@{}l@{}}{%
\textbf{}} \\
Less than high school or GED degree & 1 & 0.7 & & & \\
High school or GED & 14 & 9.7 & & & \\
Some college & 39 & 27.1 & & & \\
2-Year degree & 13 & 9.0 & & & \\
College degree & 55 & 38.2 & & & \\
Advanced degree & 22 & 15.3 & & & \\
Employment & & & & & \\
\multicolumn{6}{@{}l@{}}{%
\textbf{}} \\
Employed full-time & 70 & 48.6 & & & \\
Employed part-time & 25 & 17.4 & & & \\
Full-time student & 7 & 4.9 & & & \\
Homemaker & 1 & 0.7 & & & \\
Disabled & 7 & 4.9 & & & \\
Retired & 8 & 5.6 & & & \\
Unemployed & 15 & 10.4 & & & \\
Temporarily laid off, sick leave, or maternity leave & 3 & 2.1 & & & \\
Other, not otherwise specified & 8 & 5.6 & & & \\
Personal Income & & & \$35,050 & \$32,069 & \$0-200,000 \\
Marital Status & & & & & \\
\multicolumn{6}{@{}l@{}}{%
\textbf{}} \\
Never married & 63 & 43.8 & & & \\
Married & 32 & 22.2 & & & \\
Divorced & 42 & 29.2 & & & \\
Separated & 5 & 3.5 & & & \\
Widowed & 2 & 1.4 & & & \\


\caption{\label{tbl-2}Demographic Characterization}

\tabularnewline
\end{longtable}

\textsubscript{Source:
\href{https://jjcurtin.github.io/study_messages/notebooks/mak_tables-preview.html\#cell-tbl-2}{Make
All Tables for Main Manuscript}}

\subsection{Communications}\label{communications}

Participants had an average of 26 important contacts (range 2-113) that
were contextualized with self-report information. We obtained a total of
375,912 contextualized communications across participants. Participants
had, on average, 2,610 contextualized communications (range =
109-14,225) averaging to about 33 communications per day (range 3-278).

\subsection{Model Evaluation}\label{model-evaluation}

The median posterior auROC for the full model was 0.68, with relatively
narrow 95\% CI ({[}0.64, 0.71{]}) that did not contain .5. This provides
strong evidence that the model is capturing signal in the data. The
final model retained 13 features (Figure~\ref{fig-1}). The top four were
baseline measures of abstinence confidence, having a goal of abstinence,
abstinence self-efficacy when experiencing negative affect, and craving.
Communication frequency with people unaware of the individual's recovery
goals also emerged as an important feature associated with increased
lapse risk.

We evaluated a comparison model to assess the incremental predictive
value of cellular communication features beyond baseline measures. The
baseline model achieved comparative performance to the full model
(median auROC = 0.68, 95\% CI {[}0.64, 0.71{]}). The median difference
in auROC between the full and baseline models was less than .01,
providing no evidence (52\% probability) that the full model performed
better than the baseline model.

\begin{figure}[H]

\centering{

\pandocbounded{\includegraphics[keepaspectratio]{index_files/figure-latex/notebooks-mak_figures-fig-1-output-1.png}}

}

\caption{\label{fig-1}Global feature importance (elastic net
coefficient) for the full model. Features are ordered by absolute
coefficient value. Bars with positive coefficient values represent
features that, on average, lower lapse risk. Bars with negative
coefficient values represent features that, on average, increase risk.
Baseline features were collected via self-report measures at the
screening and intake visits. Communication features were engineered from
the contextualized cellular communications.}

\end{figure}%

\textsubscript{Source:
\href{https://jjcurtin.github.io/study_messages/notebooks/mak_figures-preview.html\#cell-fig-1}{Make
All Figures for Main Manuscript}}

\section{Discussion}\label{discussion}

Our model achieved fair performance, with an auROC of 0.68, indicating
that some predictive signal was present. However, it did not offer
incremental value beyond a baseline model that included only demographic
and self-report measures. Consistent with this, the four most important
predictors in our model were all self-report variables: abstinence
confidence, abstinence goal, negative affect efficacy, and craving.

Nonetheless, several communication features were retained in the final
model with moderately sized coefficients. These included communications
with people unaware of the participant's recovery status, non-drinkers,
friends, and individuals who were unpleasant to interact with. In
contrast, raw counts of calls and text messages and call durations were
not retained in the final model. This implies that the quantity of
communication may be less informative than the quality and social
significance. Future research may benefit from collecting richer
contextual data about communication contacts to better understand the
social dynamics contributing to lapse risk.

Even with highly contextualized communication data, however, prediction
may be limited by data sparsity. Many participants had few daily
communications, and some had extended periods with no recorded
interactions at all. Our study design may have further contributed to
this limitation. We collected only phone and SMS text communications
through the native smartphone app. In recent years, many individuals use
private messaging apps (e.g., WhatsApp, Signal) or social media
platforms (e.g., Facebook Messenger, Instagram) as their primary
communication method {[}20{]}. Therefore, our dataset likely missed a
substantial portion of participants' communications. Future studies
could explore whether incorporating communication data from additional
platforms yields stronger predictive signal. However, even with improved
data collection methods, sparsity may remain a challenge, as some people
may simply not communicate frequently with others and others use
services (e.g., Snapchat) that automatically delete messages.

We cannot entirely dismiss the potential value of cellular communication
data for risk prediction. For example, researchers have successfully
incorporated communication data into models with other sensing data
(e.g., accelerometer, geolocation, and device usage) to detect current
{[}21{]} and predict future {[}22{]} heavy drinking episondes in
non-treatment seeking young adult populations. It is possible that in
certain populations cellular communications may hold more signal. Young
adults may have more frequent communication reducing sparsity concerns.
Additionally, non-treatment seeking populations may be less likely to
sensor their data (i.e., deleting communications) when the drinking
behavior is not at odds with their goals and/or values. However, even in
these instances, the unique contribution of cellular communications
beyond other sensing methods is unclear. Some communication features,
such as outgoing call duration and the number of outgoing calls emerged
in the top 20 important features for detecting current drinking
episodes. Conversely, when predicting future drinking episodes, no
communication features appeared in the top 20. Other sensing methods,
like geolocation and accelerometer data, appeared to be more robustly
important for both detection and prediction.

Other practical challenges in collecting call and text message data
further limit the feasibility of this sensing method. For example, we
obtained participants' cellular communication data by downloading
backups of their communication logs in person during their monthly
follow-up visits. It is possible to collect cellular communication data
in real time using apps installed on Android devices. However, Apple
heavily restricts apps in its app store from accessing call and text
message data, making real-time sensing of communications challenging (if
not impossible) for IOS users. We conclude that other forms of social
interaction characterization (e.g., engineering time spent with
supportive contacts from geolocation data) are more worthwhile to pursue
in future research.

\section{Acknowledgments}\label{acknowledgments}

The authors wish to thank Susan E. Wanta for her role as the project
administrator.

\section{Author Contributions}\label{author-contributions}

KW contributed to conceptualization, data curation, formal analysis,
methodology, visualization, writing -- original draft, and writing -
review and editing. CY contributed to conceptualization, data curation,
methodology, writing - original draft. JJC contributed to
conceptualization, data curation, methadology, writing -- review and
editing, supervision of analysis, funding acquisition.

\section{Declaration of Conflicting
Interest}\label{declaration-of-conflicting-interest}

The authors declared no potential conflicts of interest with respect to
the research, authorship, and publication of this article.

\section{Funding}\label{funding}

The authors disclosed receipt of the following financial support for the
research, authorship, and publication of this article: This work was
supported by the National Institute on Alcohol Abuse and Alcoholism
(NIAAA; grant number R01 AA024391 to John J. Curtin) and the National
Institute on Drug Abuse (NIDA; grant number R01 DA047315 to John J.
Curtin).

\section{Data Availability}\label{data-availability}

Our de-identified data, questionnaires, and other study materials are
publicly available on our OSF page (\url{https://osf.io/wgpz9/}).

\newpage

\section*{References}\label{references}
\addcontentsline{toc}{section}{References}

\phantomsection\label{refs}
\begin{CSLReferences}{0}{1}
\bibitem[\citeproctext]{ref-mclellanDrugDependenceChronic2000}
\CSLLeftMargin{1. }%
\CSLRightInline{McLellan AT, Lewis DC, O'Brien CP, Kleber HD. Drug
dependence, a chronic medical illness: Implications for treatment,
insurance, and outcomes evaluation. JAMA. 2000;284: 1689--1695.
doi:\href{https://doi.org/10.1001/jama.284.13.1689}{10.1001/jama.284.13.1689}}

\bibitem[\citeproctext]{ref-dennisManagingAddictionChronic2007}
\CSLLeftMargin{2. }%
\CSLRightInline{Dennis M, Scott CK.
\href{https://www.ncbi.nlm.nih.gov/pmc/articles/PMC2797101}{Managing
{Addiction} as a {Chronic Condition}}. Addiction Science \& Clinical
Practice. 2007;4: 45--55. }

\bibitem[\citeproctext]{ref-rounsavilleLapseRelapseChasing2010}
\CSLLeftMargin{3. }%
\CSLRightInline{Rounsaville DB. Lapse, {Relapse}, and {Chasing} the
{Wagon}: {Post-Treatment Drinking} and {Recovery}. PhD thesis,
University of Maryland, Baltimore County. 2010. }

\bibitem[\citeproctext]{ref-marlattRelapsePreventionMaintenance1985}
\CSLLeftMargin{4. }%
\CSLRightInline{Marlatt GA, Gordon JR, editors. Relapse {Prevention}:
{Maintenance Strategies} in the {Treatment} of {Addictive Behaviors}.
First edition. New York: The Guilford Press; 1985. }

\bibitem[\citeproctext]{ref-marlattRelapsePreventionSecond2007}
\CSLLeftMargin{5. }%
\CSLRightInline{Marlatt GA, Donovan DM, editors. Relapse {Prevention},
{Second Edition}: {Maintenance Strategies} in the {Treatment} of
{Addictive Behaviors}. 2nd edition. New York London: The Guilford Press;
2007. }

\bibitem[\citeproctext]{ref-daleyReducingRiskRelapse2019}
\CSLLeftMargin{6. }%
\CSLRightInline{Daley DC, Douaihy A. Reducing the {Risk} of {Relapse}.
In: Daley DC, Douaihy AB, Daley DC, Douaihy A, editors. Managing
{Substance Use Disorder}: {Practitioner Guide}. Oxford University Press;
2019. p. 0.
doi:\href{https://doi.org/10.1093/med-psych/9780190926717.003.0018}{10.1093/med-psych/9780190926717.003.0018}}

\bibitem[\citeproctext]{ref-clevelandRecoveryRecoveryCapital2021}
\CSLLeftMargin{7. }%
\CSLRightInline{Cleveland HH, Brick TR, Knapp KS, Croff JM. Recovery and
{Recovery Capital}: {Aligning Measurement} with {Theory} and {Practice}.
In: Croff JM, Beaman J, editors. Family {Resilience} and {Recovery} from
{Opioids} and {Other Addictions}. Cham: Springer International
Publishing; 2021. pp. 109--128.
doi:\href{https://doi.org/10.1007/978-3-030-56958-7_6}{10.1007/978-3-030-56958-7\_6}}

\bibitem[\citeproctext]{ref-mohrPersonalSensingUnderstanding2017}
\CSLLeftMargin{8. }%
\CSLRightInline{Mohr DC, Zhang M, Schueller SM. Personal {Sensing}:
{Understanding Mental Health Using Ubiquitous Sensors} and {Machine
Learning}. Annual Review of Clinical Psychology. 2017;13: 23--47.
doi:\href{https://doi.org/10.1146/annurev-clinpsy-032816-044949}{10.1146/annurev-clinpsy-032816-044949}}

\bibitem[\citeproctext]{ref-chihPredictiveModelingAddiction2014}
\CSLLeftMargin{9. }%
\CSLRightInline{Chih M-Y, Patton T, McTavish FM, Isham AJ,
Judkins-Fisher CL, Atwood AK, et al. Predictive modeling of addiction
lapses in a mobile health application. Journal of Substance Abuse
Treatment. 2014;46: 29--35.
doi:\href{https://doi.org/10.1016/j.jsat.2013.08.004}{10.1016/j.jsat.2013.08.004}}

\bibitem[\citeproctext]{ref-wyantMachineLearningModels2024}
\CSLLeftMargin{10. }%
\CSLRightInline{Wyant K, Sant'Ana SJ, Fronk GE, Curtin JJ. Machine
learning models for temporally precise lapse prediction in alcohol use
disorder. Journal of Psychopathology and Clinical Science. 2024;133:
527--540.
doi:\href{https://doi.org/10.1037/abn0000901}{10.1037/abn0000901}}

\bibitem[\citeproctext]{ref-wyantForecastingRiskAlcoholunderreview}
\CSLLeftMargin{11. }%
\CSLRightInline{Wyant K, Fronk GE, Yu C, Punturieri CE, Curtin JJ.
Forecasting {Risk} of {Alcohol Lapse} up to {Two Weeks} in {Advance}
using {Time-lagged Machine Learning Models}. under review. }

\bibitem[\citeproctext]{ref-jonesComplianceEcologicalMomentary2019}
\CSLLeftMargin{12. }%
\CSLRightInline{Jones A, Remmerswaal D, Verveer I, Robinson E, Franken
IHA, Wen CKF, et al. Compliance with ecological momentary assessment
protocols in substance users: A meta-analysis. Addiction (Abingdon,
England). 2019;114: 609--619.
doi:\href{https://doi.org/10/gfsjzg}{10/gfsjzg}}

\bibitem[\citeproctext]{ref-wyantAcceptabilityPersonalSensing2023}
\CSLLeftMargin{13. }%
\CSLRightInline{Wyant K, Moshontz H, Ward SB, Fronk GE, Curtin JJ.
Acceptability of {Personal Sensing Among People With Alcohol Use
Disorder}: {Observational Study}. JMIR mHealth and uHealth. 2023;11:
e41833. doi:\href{https://doi.org/10.2196/41833}{10.2196/41833}}

\bibitem[\citeproctext]{ref-aczelConsensusbasedTransparencyChecklist2019}
\CSLLeftMargin{14. }%
\CSLRightInline{Aczel B, Szaszi B, Sarafoglou A, Kekecs Z, Kucharský Š,
Benjamin D, et al. A consensus-based transparency checklist. Nature
Human Behaviour. 2019; 1--3.
doi:\href{https://doi.org/10.1038/s41562-019-0772-6}{10.1038/s41562-019-0772-6}}

\bibitem[\citeproctext]{ref-hsiehSampleSizeTables1989}
\CSLLeftMargin{15. }%
\CSLRightInline{Hsieh F. Sample size tables for logistic regression.
Statistics in Medicine. 1989;8: 795--802. }

\bibitem[\citeproctext]{ref-derogatislBriefSymptomInventory}
\CSLLeftMargin{16. }%
\CSLRightInline{Derogatis, L.R. Brief {Symptom Inventory} 18 -
{Administration}, scoring, and procedures manual. Minneapolis: NCS
Pearson; 2000. }

\bibitem[\citeproctext]{ref-hurlbutAssessingAlcoholProblems1992}
\CSLLeftMargin{17. }%
\CSLRightInline{Hurlbut SC, Sher KJ. Assessing alcohol problems in
college students. Journal of American College Health. 1992;41: 49--58. }

\bibitem[\citeproctext]{ref-mckiernanDevelopmentBriefAbstinence2011}
\CSLLeftMargin{18. }%
\CSLRightInline{McKiernan P, Cloud R, Patterson DA, Wolf S, Golder S,
Besel K. Development of a {Brief Abstinence Self-Efficacy Measure}.
Journal of social work practice in the addictions. 2011;11: 245--253.
doi:\href{https://doi.org/10.1080/1533256X.2011.593445}{10.1080/1533256X.2011.593445}}

\bibitem[\citeproctext]{ref-flanneryPsychometricPropertiesPenn1999}
\CSLLeftMargin{19. }%
\CSLRightInline{Flannery BA, Volpicelli JR, Pettinati HM.
\href{https://www.ncbi.nlm.nih.gov/pubmed/10470970}{Psychometric
properties of the {Penn Alcohol Craving Scale}}. Alcoholism, clinical
and experimental research. 1999;23: 1289--1295. }

\bibitem[\citeproctext]{ref-mcdowellPreferencesAttitudesDigital2025}
\CSLLeftMargin{20. }%
\CSLRightInline{McDowell B, Dumais KM, Gary ST, de Gooijer I, Ward T.
Preferences and {Attitudes Towards Digital Communication} and {Symptom
Reporting Methods} in {Clinical Trials}. Patient preference and
adherence. 2025;19: 255--263.
doi:\href{https://doi.org/10.2147/PPA.S474535}{10.2147/PPA.S474535}}

\bibitem[\citeproctext]{ref-baeDetectingDrinkingEpisodes2017}
\CSLLeftMargin{21. }%
\CSLRightInline{Bae S, Ferreira D, Suffoletto B, Puyana JC, Kurtz R,
Chung T, et al. Detecting {Drinking Episodes} in {Young Adults Using
Smartphone-based Sensors}. Proceedings of the ACM on Interactive,
Mobile, Wearable and Ubiquitous Technologies. 2017;1: 1--36.
doi:\href{https://doi.org/10.1145/3090051}{10.1145/3090051}}

\bibitem[\citeproctext]{ref-baeLeveragingMobilePhone2023}
\CSLLeftMargin{22. }%
\CSLRightInline{Bae SW, Suffoletto B, Zhang T, Chung T, Ozolcer M, Islam
MR, et al. Leveraging {Mobile Phone Sensors}, {Machine Learning} and
{Explainable Artificial Intelligence} to {Predict Imminent Same-Day
Binge Drinking Events} to {Support Just-In-Time Adaptive Interventions}:
{A Feasibility Study}. JMIR formative research. 2023.
doi:\href{https://doi.org/10.2196/39862}{10.2196/39862}}

\end{CSLReferences}




\end{document}
